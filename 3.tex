\chapter{线性空间}

%——————————————————————————————————%

\section{线性空间}

\begin{definition}{数域}
  设 $\mathbb{K}$ 是复数集 $\mathbb{C}$ 的子集,且至少是两个不同的元素。若 $\mathbb{K}$ 中任意两个数的加法、减法、乘法下封闭(除数不为零)仍属于 $\mathbb{K}$,则称 $\mathbb{K}$ 是一个数域。
\end{definition}

\begin{theorem}
  任一数域必定包含有理数域 $\mathbb{Q}$。
\end{theorem}

\begin{definition}{向量}
  设数域 $\mathbb{K}$,$a_1,a_2, \ldots ,a_n$ 是 $\mathbb{K}$ 中的元素。由 $a_1,a_2, \ldots ,a_n$ 组成的有序数组 $(a_1,a_2, \ldots ,a_n)$ 称为 $\mathbb{K}$ 上的 $n$ 维行向量。有序数组
  \[
    \begin{pmatrix}
      a_1    \\
      a_2    \\
      \vdots \\
      a_n
    \end{pmatrix}
  \]
  称为 $\mathbb{K}$ 上的 $n$ 维列向量。为了方便,$n$ 维列向量常记为 $n$ 维行向量的转置形式:$(a_1,a_2, \ldots a_n)^{\mathrm{T}}$。

  行向量和列向量统称为向量。
\end{definition}

\begin{remark}
  $n$ 维行向量可以看成 $1 \times n$ 矩阵,$n$ 维列向量可以看成 $n \times 1$ 矩阵。而且向量运算规则与矩阵的相应运算规则相同。
\end{remark}

\begin{proposition}
  两个向量相等当且仅当它们的元素相同,且元素出现的次序也相同。
\end{proposition}

\begin{definition}{零向量}
  若一个 $n$ 维向量的所有元素都为零,则称它为零向量,记为 $\bm{0}$。
\end{definition}

\begin{definition}{负向量}
  令向量 $\bm{\alpha} = (a_1,a_2, \ldots ,a_n)$。则记 $\bm{\alpha}$ 的负向量 $-\bm{\alpha} = -(a_1,a_2, \ldots ,a_n)$。
\end{definition}

\begin{proposition}{向量运算规则}
  \begin{enumerate}
    \item $\bm{\alpha} + \bm{\beta} = \bm{\beta} + \bm{\alpha}$;
    \item $(\bm{\alpha} + \bm{\beta}) + \bm{\gamma} = \bm{\alpha} + (\bm{\beta} + \bm{\gamma})$;
    \item $\bm{\alpha} + \bm{0} = \bm{\alpha}$;
    \item $\bm{\alpha} + (-\bm{\alpha}) = \bm{0}$;
    \item $1 \cdot\bm{\alpha}= \bm{\alpha}$;
    \item $k(\bm{\alpha} + \bm{\beta}) = k\bm{\alpha} + k\bm{\beta}, k \in \mathbb{K}$;
    \item $(k + l)\bm{\alpha} = k\bm{\alpha} + l\bm{\alpha}$;
    \item $k(l\bm{\alpha}) = (kl)\bm{\alpha}$。
  \end{enumerate}
\end{proposition}
\begin{remark}
  二维实向量加法的几何意义是平行四边形法则;数乘的几何意义是原向量的同向或反向的倍数。
\end{remark}

\begin{definition}
  数域 $\mathbb{K}$ 上的 $n$ 维行向量全体组成的集合称为 $\mathbb{K}$ 上的 $n$ 维行向量空间。
  数域 $\mathbb{K}$ 上的 $n$ 维列向量全体组成的集合称为 $\mathbb{K}$ 上的 $n$ 维列向量空间。
\end{definition}

\begin{definition}{线性空间}
  设数域 $\mathbb{K}$,集合 $V$。若在 $V$ 上定义加法运算 “$+$”:对 $\bm{\alpha},\bm{\beta} \in V$,$\exists !\bm{\gamma}$ 与之对应,记为 $\bm{\gamma} = \bm{\alpha} + \bm{\beta}$。在 $\mathbb{K}$ 和 $V$ 之间定义数乘运算:对 $\forall k \in \mathbb{K}, \bm{\alpha}\in V$,$\exists !\bm{\delta}\in V$ 与之对应,记为 $\bm{\delta} = k\bm{\alpha}$。若上述加法和数乘满足:
  \begin{enumerate}
    \item $\bm{\alpha} + \bm{\beta} = \bm{\beta} + \bm{\alpha}$;
    \item $(\bm{\alpha} + \bm{\beta}) + \bm{\gamma} = \bm{\alpha} + (\bm{\beta} + \bm{\gamma})$;
    \item $\exists \bm{0}\in V, \forall \bm{\alpha}\in V, \bm{\alpha} + \bm{0} = \bm{\alpha}$;
    \item $\forall \bm{\alpha} \in V, \exists \bm{\beta} \in V, \bm{\alpha} + (-\bm{\alpha}) = \bm{0}$;
    \item $1 \cdot\bm{\alpha}= \bm{\alpha}$;
    \item $k(\bm{\alpha} + \bm{\beta}) = k\bm{\alpha} + k\bm{\beta}$;
    \item $(k + l)\bm{\alpha} = k\bm{\alpha} + l\bm{\alpha}$;
    \item $k(l\bm{\alpha}) = (kl)\bm{\alpha}$。
  \end{enumerate}
  其中 $\bm{\alpha},\bm{\beta},\bm{\gamma}$ 是 $V$ 中任意元素,$k,l$ 是 $\mathbb{K}$ 中任意元素,则集合 $V$ 称为 $\mathbb{K}$ 上的线性空间或向量空间。$V$ 中的元素称为向量,$V$ 中满足(3)的元素 $\bm{0}$ 称为零向量,满足(4)的 $\bm{\beta}$ 称为 $\bm{\alpha}$ 的负向量,记为 $-\bm{\alpha}$。
\end{definition}

\hfill

\begin{example}
  数域 $\mathbb{K}$ 上的 $n$ 维行向量集合(列向量集合)是 $\mathbb{K}$ 上的线性空间,记为 $\mathbb{K}^n(\mathbb{K}_n)$。
\end{example}

\hfill

\begin{example}
  系数取自数域 $\mathbb{K}$ 的一元多项式的全体,记为 $\mathbb{K}[x]$,则 $\mathbb{K}[x]$ 是 $\mathbb{K}$ 上的线性空间。

  在 $\mathbb{K}[x]$ 中取次数小于等于 $n$ 的多项式全体,记为 $\mathbb{K}_n[x]$,则 $\mathbb{K}_n[x]$ 也是 $\mathbb{K}$ 上的线性空间。
\end{example}

\hfill

\begin{example}
  复数域 $\mathbb{C}$ 可以看成实数域 $\mathbb{R}$ 上的线性空间。
\end{example}

\begin{remark}
  一般地,若两个数域 $\mathbb{K}_1 \subseteq \mathbb{K}_2$,则 $\mathbb{K}_2$ 可以看成 $\mathbb{K}_1$ 上的线性空间。特别地,$\mathbb{K}$ 可以看成 $\mathbb{K}$ 自身上的线性空间。
\end{remark}

\begin{proposition}
  零向量唯一。
\end{proposition}

\begin{proposition}
  负向量唯一。
\end{proposition}

\begin{proposition}
  对 $\forall \bm{\alpha},\bm{\beta},\bm{\gamma} \in V$,有
  \begin{enumerate}
    \item $\bm{\alpha} + \bm{\beta} = \bm{\alpha} + \bm{\gamma} \implies \bm{\beta} = \bm{\gamma}$;
    \item $0\cdot\bm{\alpha} = \bm{0}$;
    \item $k\cdot\bm{0} = \bm{0}$;
    \item $(-1)\bm{\alpha} = -\bm{\alpha}$;
    \item $k\bm{\alpha} = 0 \implies \bm{\alpha} = 0$ 或 $k = 0$。
  \end{enumerate}
\end{proposition}

\begin{proposition}{线性空间相等}
  两个线性空间 $V$ 和 $U$ 相等当且仅当 $V$ 的基和 $U$ 的基可以互相线性表示。
\end{proposition}

\begin{remark}
  线性空间本质是集合,因此证明两个线性空间相等可以证明二者互相包含。
\end{remark}



%——————————————————————————————————%

\section{向量的线性关系}






%——————————————————————————————————%

\section{向量组的秩}






%——————————————————————————————————%

\section{矩阵的秩}






%——————————————————————————————————%

\section{坐标向量}






%——————————————————————————————————%

\section{基变换与过渡矩阵}






%——————————————————————————————————%

\section{子空间}






%——————————————————————————————————%

\section{线性方程组的解}






%——————————————————————————————————%
