\chapter{内积空间}

%——————————————————————————————————%

\section{内积空间}

\begin{definition}
    设 $V$ 是实数域上的线性空间,若对 $V$ 中任意一组有序向量 $\{\bm{\alpha}, \bm{\beta}\}$,都唯一地对应一个实数,记为 $\langle \bm{\alpha}, \bm{\beta} \rangle$,满足下列条件:
    \begin{enumerate}
        \item $\langle \bm{\beta}, \bm{\alpha} \rangle = \langle \bm{\alpha}, \bm{\beta} \rangle$;
        \item $\langle \bm{\alpha} + \bm{\beta}, \bm{\gamma} \rangle = \langle \bm{\alpha}, \bm{\gamma} \rangle + \langle \bm{\beta}, \bm{\gamma} \rangle$;
        \item $\langle c\bm{\alpha}, \bm{\beta} \rangle = c\langle \bm{\alpha}, \bm{\beta} \rangle$,其中 $c$ 为任一实数;
        \item $\langle \bm{\alpha}, \bm{\alpha} \rangle \geqslant 0$ 且等号成立当且仅当 $\bm{\alpha} = \bm{0}$。
    \end{enumerate}
    则称在 $V$ 上定义了一个内积,实数 $\langle \bm{\alpha}, \bm{\beta} \rangle$ 称为 $\bm{\alpha}$ 与 $\bm{\beta}$ 的内积。线性空间 $V$ 称为实内积空间。有限维实内积空间称为 Euclid 空间,简称欧氏空间。
\end{definition}

\begin{definition}
    设 $V$ 是复数域上的线性空间,若对 $V$ 中任意一组有序向量 $\{\bm{\alpha}, \bm{\beta}\}$,都唯一地对应一个复数,记为 $\langle \bm{\alpha}, \bm{\beta} \rangle$,满足下列条件:
    \begin{enumerate}
        \item $\langle \bm{\beta}, \bm{\alpha} \rangle = \overline{\langle \bm{\alpha}, \bm{\beta} \rangle}$;
        \item $\langle \bm{\alpha} + \bm{\beta}, \bm{\gamma} \rangle = \langle \bm{\alpha}, \bm{\gamma} \rangle + \langle \bm{\beta}, \bm{\gamma} \rangle$;
        \item $\langle c\bm{\alpha}, \bm{\beta} \rangle = c\langle \bm{\alpha}, \bm{\beta} \rangle$,其中 $c$ 为任一复数;
        \item $\langle \bm{\alpha}, \bm{\alpha} \rangle \geqslant 0$ 且等号成立当且仅当 $\bm{\alpha} = \bm{0}$。
    \end{enumerate}
    则称在 $V$ 上定义了一个内积,复数 $\langle \bm{\alpha}, \bm{\beta} \rangle$ 称为 $\bm{\alpha}$ 与 $\bm{\beta}$ 的内积。线性空间 $V$ 称为复内积空间。有限维复内积空间称为酉空间。
\end{definition}

\begin{remark}
    由 $(1)$ 和 $(3)$,有 $\langle \bm{\alpha}, c\bm{\beta} \rangle = \overline{c}\langle \bm{\alpha}, \bm{\beta} \rangle$。
\end{remark}

\begin{definition}
    设 $\mathbb{R}_n$ 是 $n$ 维实列向量空间,$\bm{\alpha} = (x_1, x_2, \ldots, x_n)^{\mathrm{T}}, \bm{\beta} = (y_1, y_2, \ldots, y_n)^{\mathrm{T}}$,定义
    \[
        \langle \bm{\alpha}, \bm{\beta} \rangle = x_{1}y_{1} + x_{2}y_{2} + \cdots + x_{n}y_{n}
    \]
    则 $\mathbb{R}_n$ 成为一个欧式空间,上述内积称为 $\mathbb{R}_n$ 的标准内积。
\end{definition}

\begin{definition}
    设 $\mathbb{C}_n$ 是 $n$ 维复列向量空间,$\bm{\alpha} = (x_1, x_2, \ldots, x_n)^{\mathrm{T}}, \bm{\beta} = (y_1, y_2, \ldots, y_n)^{\mathrm{T}}$,定义
    \[
        \langle \bm{\alpha}, \bm{\beta} \rangle = x_{1}\overline{y}_{1} + x_{2}\overline{y}_{2} + \cdots + x_{n}\overline{y}_{n}
    \]
    则 $\mathbb{C}_n$ 成为一个酉空间,上述内积称为 $\mathbb{C}_n$ 的标准内积。
\end{definition}

\begin{proposition}
    设 $V$ 是 $n$ 维实列向量空间,$\bm{G}$ 是 $n$ 阶正定实对称阵,对 $\bm{\alpha}, \bm{\beta} \in V$,定义
    \[
        \langle \bm{\alpha}, \bm{\beta} \rangle = \bm{\alpha}^{\mathrm{T}}\bm{G\beta}
    \]
    则 $V$ 成为一个欧式空间。当 $\bm{G} = \bm{I}_n$,即 $\bm{G}$ 为单位阵时,$V$ 上的内积就是标准内积。

    实列向量空间的标准内积用矩阵乘法可表示为
    \[
        \langle \bm{\alpha}, \bm{\beta} \rangle = \bm{\alpha}^{\mathrm{T}}\bm{\beta}
    \]

    实行向量空间的标准内积用矩阵乘法可表示为
    \[
        \langle \bm{\alpha}, \bm{\beta} \rangle = \bm{\alpha}\bm{\beta}^{\mathrm{T}}
    \]
\end{proposition}

\begin{proposition}
    设 $U$ 是 $n$ 维复列向量空间,$\bm{H}$ 是 $n$ 阶正定 Hermite 阵,对 $\bm{\alpha}, \bm{\beta} \in U$,定义
    \[
        \langle \bm{\alpha}, \bm{\beta} \rangle = \bm{\alpha}^{\mathrm{T}}\bm{H}\overline{\bm{\beta}}
    \]
    则 $U$ 成为一个欧式空间。当 $\bm{H} = \bm{I}_n$,即 $\bm{H}$ 为单位阵时,$U$ 上的内积就是标准内积。

    复列向量空间的标准内积用矩阵乘法可表示为
    \[
        \langle \bm{\alpha}, \bm{\beta} \rangle = \bm{\alpha}^{\mathrm{T}}\overline{\bm{\beta}}
    \]

    复行向量空间的标准内积用矩阵乘法可表示为
    \[
        \langle \bm{\alpha}, \bm{\beta} \rangle = \bm{\alpha}\overline{\bm{\beta}}^{\mathrm{T}}
    \]
\end{proposition}

\begin{definition}
    设 $V$ 是内积空间,$\bm{\alpha}$ 是 $V$ 中的向量,定义 $\bm{\alpha}$ 的范数为
    \[
        \Vert \bm{\alpha} \Vert = \sqrt{\langle \bm{\alpha}, \bm{\alpha} \rangle}
    \]
    即实数 $\langle \bm{\alpha}, \bm{\alpha} \rangle$ 的算术平方根。
\end{definition}

\begin{remark}
    根据范数的定义,$\Vert \bm{\alpha} \Vert = 0$ 当且仅当 $\bm{\alpha} = \bm{0}$。
\end{remark}

\begin{remark}
    当 $V = \mathbb{R}^n$ 且内积为标准内积时,若 $\bm{\alpha} = (x_1, x_2, \ldots, x_n)$,则
    \[
        \Vert \bm{\alpha} \Vert = \sqrt{x_{1}^{2} + x_{2}^{2} + \cdots + x_{n}^{2}}
    \]
\end{remark}

\begin{remark}
    设 $\bm{\alpha}, \bm{\beta} \in V$,则 $\bm{\alpha}$ 与 $\bm{\beta}$ 的距离定义为
    \[
        d(\bm{\alpha}, \bm{\beta}) = \Vert \bm{\alpha} - \bm{\beta} \Vert
    \]
    显然 $d(\bm{\alpha}, \bm{\beta}) = d(\bm{\beta}, \bm{\alpha})$。
\end{remark}

\begin{theorem}
    设 $V$ 是内积空间,$\bm{\alpha}, \bm{\beta} \in V$,$c$ 是任一常数,则
    \begin{enumerate}
        \item $\Vert c \bm{\alpha} \Vert = \vert c \vert \cdot \Vert \bm{\alpha} \Vert $;
        \item $\vert \langle \bm{\alpha}, \bm{\beta} \rangle \vert \leqslant \Vert \bm{\alpha} \Vert \cdot \Vert \bm{\beta} \Vert $;
        \item $\Vert \bm{\alpha} + \bm{\beta} \Vert \leqslant \Vert \bm{\alpha} \Vert + \Vert \bm{\beta} \Vert $。
    \end{enumerate}
\end{theorem}

\begin{remark}
    我们将 $(2)$ 称为 Cauchy-Schuwarz 不等式,$(3)$ 称为三角不等式。
\end{remark}

\begin{definition}
    当 $V$ 是实内积空间时,定义非零向量 $\bm{\alpha}, \bm{\beta}$ 的夹角 $\theta$ 的余弦为
    \[
        \cos \theta = \frac{\langle \bm{\alpha}, \bm{\beta} \rangle}{\Vert \bm{\alpha} \Vert \cdot \Vert \bm{\beta} \Vert}
    \]
    当 $V$ 是复内积空间时,定义非零向量 $\bm{\alpha}, \bm{\beta}$ 的夹角 $\theta$ 的余弦为
    \[
        \cos \theta = \frac{\vert \langle \bm{\alpha}, \bm{\beta} \rangle \vert}{\Vert \bm{\alpha} \Vert \cdot \Vert \bm{\beta} \Vert}
    \]
    若内积空间中的两个向量 $\bm{\alpha}, \bm{\beta}$ 满足 $\langle \bm{\alpha}, \bm{\beta} \rangle = 0$,则称 $\bm{\alpha}$ 与 $\bm{\beta}$ 垂直或正交,记为 $\bm{\alpha} \perp \bm{\beta}$。
\end{definition}

\begin{remark}
    正交的性质:
    \begin{enumerate}
        \item 零向量与任何向量都正交;
        \item 若 $\bm{\alpha}$ 与 $\bm{\beta}$ 正交,则 $\bm{\beta}$ 也与 $\bm{\alpha}$ 正交;
        \item 两个非零向量 $\bm{\alpha}, \bm{\beta}$ 正交时夹角为 $90^{\circ}$。
    \end{enumerate}
\end{remark}

\begin{remark}
    若 $\bm{\alpha}$ 与 $\bm{\beta}$ 正交,则 $\langle \bm{\alpha}, \bm{\beta} \rangle = \langle \bm{\beta}, \bm{\alpha} \rangle = 0$,于是有勾股定理 ${\Vert \bm{\alpha} + \bm{\beta} \Vert}^{2} = {\Vert \bm{\alpha} \Vert}^{2} + {\Vert \bm{\beta} \Vert}^{2}$;
\end{remark}

\begin{remark}
    设 $V$ 是 $n$ 维实行向量空间,内积取标准内积,则得 Cauchy 不等式
    \[
        (x_{1}y_{1} + x_{2}y_{2} + \cdots + x_{n}y_{n})^{2} \leqslant  (x_{1}^{2} + x_{2}^{2} + \cdots + x_{n}^{2})(y_{1}^{2} + y_{2}^{2} + \cdots + y_{n}^{2})
    \]
\end{remark}

\begin{remark}
    设 $V$ 是由 $[a, b]$ 区间上的连续函数全体构成的实线性空间,内积定义为
    \[
        \langle f, g \rangle = \int_{a}^{b}f(t)g(t)\mathrm{d}t
    \]
    则得 Schuwarz 不等式
    \[
        \left(\int_{a}^{b}f(t)g(t)\mathrm{d}t\right)^{2} \leqslant \int_{a}^{b}f(t)^{2}\mathrm{d}t \int_{a}^{b}g(t)^{2}\mathrm{d}t
    \]
\end{remark}
%——————————————————————————————————%

\section{}






%——————————————————————————————————%

\section{}






%——————————————————————————————————%

\section{}






%——————————————————————————————————%

\section{}






%——————————————————————————————————%

\section{}






%——————————————————————————————————%

\section{}






%——————————————————————————————————%

\section{}






%——————————————————————————————————%

\section{}






%——————————————————————————————————%

\section{}






%——————————————————————————————————%

\section{}






%——————————————————————————————————%

\section{}






