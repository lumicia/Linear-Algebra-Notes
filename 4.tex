\chapter{线性映射}

%——————————————————————————————————%

\section{线性映射的概念}

\begin{definition}{线性映射}
    设数域 $\mathbb{K}$ 上的线性空间 $V,U$。令映射 $\varphi : V \to U$。若
    \begin{enumerate}
        \item $\varphi(\alpha + \beta) = \varphi(\alpha) + \varphi(\beta),\, \alpha,\beta \in V$;
        \item $\varphi(k\alpha) = k\varphi(\alpha),\, k \in \mathbb{K},\alpha \in V$。
    \end{enumerate}
    则称 $\varphi$ 是 $V \to U$ 的线性映射。

    若 $\varphi V \to V$,则称 $\varphi$ 为线性变换。

    若映射 $\varphi$ 是单的,则称 $\varphi$ 为单线性映射。

    若映射 $\varphi$ 是满的,则称 $\varphi$ 为满线性映射。

    若映射 $\varphi$ 是双射,则称 $\varphi$ 为线性同构,简称同构。若 $V = U$,则 $V$ 自身上的同构称为自同构。
\end{definition}

\hfill

\begin{example}
    零映射:$\forall \alpha \in V, \varphi(\alpha) = \mathbf{0}$。记为 $\mathbf{0}$。
\end{example}

\hfill

\begin{example}
    恒等变换:恒等映射 $\mathbf{1}_V$ 是 $V$ 上的线性变换,记为 $I_V$,简记为 $I$。
\end{example}

\hfill

\begin{example}
    设数域 $\mathbb{K}$ 上的列向量空间 $V = \mathbb{K}_n, U = \mathbb{K}_m$ 和 $m \times n$ 矩阵 $A$,映射 $\varphi : V \to U$ 定义为
    \[\varphi(\alpha) = A\alpha\]

    $\varphi$ 由矩阵乘法定义,可得 $\varphi$ 是线性映射。
\end{example}

\hfill

\begin{proposition}
    设线性映射 $\varphi : V \to U$,则
    \begin{enumerate}
        \item $\varphi(0) = 0$;
        \item $\varphi(k\alpha + l\beta) = k\varphi(\alpha) + l\varphi(\beta),\, \alpha,\beta \in V,\, k,l \in \mathbb{K}$;
        \item 若 $\varphi$ 是同构,则 $\varphi$ 的逆映射 $\varphi^{-1}$ 也是线性映射,从而是 $U \to V$ 的同构。
    \end{enumerate}
\end{proposition}



%——————————————————————————————————%

\section{线性映射的运算}

\begin{definition}
    设 $\varphi,\psi$ 是 $\mathbb{K}$ 上线性空间 $V \to U$ 的线性映射。定义 $\varphi + \psi$ 为 $V \to U$ 的映射:
    \[(\varphi + \psi)(\alpha) = \varphi(\alpha) + \psi(\alpha),\, \alpha \in  V\]
    若 $k \in \mathbb{K}$,定义 $k\varphi$ 为 $V \to U$ 的映射:
    \[(k\varphi)(\alpha) = k\varphi(\alpha),\, \alpha \in V\]
\end{definition}

\begin{proposition}
    上面定义的映射 $\varphi + \psi$ 和 $k\varphi$ 都是线性映射。
\end{proposition}

\begin{proposition}
    设 $\mathcal{L}(V,U)$ 是 $V \to U$ 的线性映射的全体,则在线性映射 $\varphi + \psi$ 和 $k\varphi$ 下,$\mathcal{L}(V,U)$ 是 $\mathbb{K}$ 上的线性空间。

    特别地,$V \to \mathbb{K}$ 的所有线性函数全体构成一个线性空间,称为 $V$ 的共轭空间,记为 $V^*$。当 $V$ 是有限维空间时,$V^*$ 称为 $V$ 的对偶空间。
\end{proposition}

\begin{remark}
    若 $U = \mathbb{K}$,即把 $\mathbb{K}$ 看成 $\mathbb{K}$ 上的一维空间,则 $V \to \mathbb{K}$ 的线性映射称为线性函数。
\end{remark}

\begin{remark}
  若 $V = U$,则 $\mathcal{L}(V,V)$ 是 $V$ 上线性变换的全体构成的线性空间,简记为 $\mathcal{L}(V)$。$\mathcal{L}(V)$ 上还有乘法运算,即映射的复合。
\end{remark}

\begin{definition}
  设 $A$ 是 $\mathbb{K}$ 上的线性空间。若在 $A$ 上定义了乘法 $\cdot$(可省略),使得对 $\forall a,b,c \in A$,及 $k \in \mathbb{K}$,满足
  \begin{enumerate}
    \item $a(bc) = (ab)c$;
    \item $\exists e \in A,\, ea = ae = a$;
    \item $a(b + c) = ab + ac,\, (b + c)a = ba + ca$;
    \item $(ka)b = k(ab) = a(kb)$。
  \end{enumerate}
  则称 $A$ 是 $\mathbb{K}$ 上的代数,$e$ 称为 $A$ 的幺元。
\end{definition}

\begin{theorem}
  设 $V$ 是 $\mathbb{K}$ 上的线性空间,则 $\mathcal{L}(V)$ 是 $\mathbb{K}$ 上的代数。
\end{theorem}

\begin{proposition}
  在 $\mathcal{L}(V)$ 中,定义线性变换 $\varphi$ 的 $n$ 次幂为 $n$ 个 $\varphi$ 的复合,则
  \[\varphi^n \circ \varphi^m = \varphi^{n + m},\, (\varphi^n)^m = \varphi^{nm}\]
  若 $\varphi$ 是双射,即 $\varphi$ 是 $V$ 上的自同构,则 $\varphi^{-1}$ 也是 $V$ 上的线性变换,而且也是自同构,称 $\varphi^{-1}$ 是 $\varphi$ 的逆变换。定义
  \[\varphi^{-n} = (\varphi^{-1})^n\]
  则
  \[\varphi^{-n} = (\varphi^n)^{-1}\]
  定义
  \[\varphi^0 = I_V\]
  $\varphi$ 的负数次幂仅对自同构(又称可逆变换或非异变换)有意义。
\end{proposition}

\begin{proposition}
  若 $\varphi$ 和 $\psi$ 都是可逆线性变换,则 $\varphi \circ \psi$ 也是可逆线性变换,且
  \[(\varphi \circ \psi)^{-1} = \psi^{-1} \circ \varphi^{-1}\]
  对任意 $k \ne 0$,若 $\varphi$ 可逆,则 $k\varphi$ 也可逆,且
  \[(k\varphi)^{-1} = k^{-1}\varphi^{-1}\]
\end{proposition}


%——————————————————————————————————%

\section{线性映射与矩阵}






%——————————————————————————————————%

\section{线性映射的核与像}






%——————————————————————————————————%

\section{不变子空间}






%——————————————————————————————————%

\section{}






%——————————————————————————————————%

\section{}






%——————————————————————————————————%

\section{}






%——————————————————————————————————%

\section{}






%——————————————————————————————————%

\section{}






%——————————————————————————————————%

\section{}






%——————————————————————————————————%

\section{}






