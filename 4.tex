\chapter{线性映射}

%——————————————————————————————————%

\section{线性映射的概念}

\begin{definition}
  若有集合 $A$ 到集合 $B$ 的一个对应 $\varphi : A \to B$,则对 $\forall a \in A$,均有唯一的 $b \in B$ 与 $a$ 对应,记为 $b = \varphi(a)$。元素 $b$ 称为 $a$ 在 $\varphi$ 下的像,$a$ 称为 $b$ 的原像或逆像。$A$ 中元素在 $\varphi$ 下的像全体构成 $B$ 的子集,记为 $\varphi(a)$ 或 $\im(\varphi)$。

  若 $\im(\varphi) = B$,即对 $\forall b \in B$,在 $A$ 中均有元素 $a$ 使得 $b = \varphi(a)$,则称 $\varphi$ 是满射或映上映射。

  若映射 $\varphi$ 满足条件:当 $a \ne a'$ 时,$\varphi(a) \ne \varphi(a')$,则称 $\varphi$ 是单射。(该条件的一个等价形式为:当 $\varphi(a) = \varphi(a')$ 时,$a = a'$。

  若 $\varphi$ 既是单射又是满射,则称 $\varphi$ 是双射或一一对应。
\end{definition}

\hfill

\begin{example}
  函数 $y = x^2$ 既不是单射也不是满射。
\end{example}

\hfill

\begin{example}
  函数 $y = x^3$ 是单射。
\end{example}

\hfill

\begin{example}
  设 $\mathbb{R}$ 上 $n(n > 1)$ 的方阵全体组成的集合 $M_n(\mathbb{R})$。映射 $\varphi : M_n(\mathbb{R}) \to \mathbb{R}$ 定义为
  \[\varphi(A) = \det A\]
  则 $\varphi$ 是满射。
\end{example}

\hfill

\begin{example}
  设 Descartes 平面上的点 $C$ 的横坐标为 $a$,纵坐标为 $b$,点到实数对之间的对应 $\varphi$ 定义为
  \[C \mapsto (a, b)\]
  则 $\varphi$ 是双射。
\end{example}

\hfill

\begin{definition}{恒等映射}
  若集合 $A \subseteq B$,映射 $j:A \to B$ 定义为
  \[j(a) = a,a \in A\]
  则 $j$ 是单射。若 $A = B$,则 $j$ 是满射,此时 $j: A \to A$,称 $j$ 是恒等映射,记为 $1_A$ 或 $I_A$。
\end{definition}

\begin{definition}{变换}
  设映射 $f: A \to A$,则称 $f$ 是一个变换。
\end{definition}

\begin{definition}{映射相等}
  设映射 $f: A \to B$ 和 $g: A \to B$。若对 $\forall a \in A$ 都有 $f(a) = g(a)$,则称 $f$ 和 $g$ 相等,记为 $f = g$。
\end{definition}

\begin{definition}{复合映射}
  设映射 $f: A \to B, g: B \to C$,定义 $g$ 与 $f$ 的复合映射 $g \circ f : A \to C$,且
  \[(g \circ f)(a) = g(f(a)),a \in A\]
\end{definition}

\begin{proposition}{映射复合的结合律}
  设映射 $f: A \to B, g: B \to C, h: C \to D$,则
  \[(h \circ g)\circ f = h \circ(g \circ f)\]
\end{proposition}

通过结合律可以省略复合映射的括号,即把 $g \circ f$ 简写为 $gf$,$h \circ(g \circ f)$ 简写为 $hgf$。

\begin{proposition}
  设映射 $f: A \to B$,若存在映射 $g: B \to A$ 使得
  \[gf = 1_A,\, fg = 1_B\]
  则 $f$ 是双射,且 $g = f^{-1}$。
\end{proposition}

$gf = 1_A \implies f$ 是单射,$fg = 1_B \implies f$ 是满射。

\begin{proposition}
  设双射 $f: A \to B,g: B \to C$,则
  \begin{enumerate}
    \item $gf: A \to C$ 也是双射;
    \item $(gf)^{-1} = f^{-1}g^{-1}$。
  \end{enumerate}
\end{proposition}

\begin{definition}{线性映射}
  设数域 $\mathbb{K}$ 上的线性空间 $V,U$。令映射 $\varphi : V \to U$。若
  \begin{enumerate}
    \item $\varphi(\alpha + \beta) = \varphi(\alpha) + \varphi(\beta),\, \alpha,\beta \in V$;
    \item $\varphi(k\alpha) = k\varphi(\alpha),\, k \in \mathbb{K},\alpha \in V$。
  \end{enumerate}
  则称 $\varphi$ 是 $V \to U$ 的线性映射。

  若 $\varphi V \to V$,则称 $\varphi$ 为线性变换。

  若映射 $\varphi$ 是单的,则称 $\varphi$ 为单线性映射。

  若映射 $\varphi$ 是满的,则称 $\varphi$ 为满线性映射。

  若映射 $\varphi$ 是双射,则称 $\varphi$ 为线性同构,简称同构。若 $V = U$,则 $V$ 自身上的同构称为自同构。
\end{definition}

\hfill

\begin{example}
  零映射:$\forall \alpha \in V, \varphi(\alpha) = \mathbf{0}$。记为 $\mathbf{0}$。
\end{example}

\hfill

\begin{example}
  恒等变换:恒等映射 $\mathbf{1}_V$ 是 $V$ 上的线性变换,记为 $I_V$,简记为 $I$。
\end{example}

\hfill

\begin{example}\label{example-1.7}
  设数域 $\mathbb{K}$ 上的列向量空间 $V = \mathbb{K}_n, U = \mathbb{K}_m$ 和 $m \times n$ 矩阵 $A$,映射 $\varphi : V \to U$ 定义为
  \[\varphi(\alpha) = A\alpha\]

  $\varphi$ 由矩阵乘法定义,可得 $\varphi$ 是线性映射。
\end{example}

\hfill

\begin{proposition}
  设线性映射 $\varphi : V \to U$,则
  \begin{enumerate}
    \item $\varphi(0) = 0$;
    \item $\varphi(k\alpha + l\beta) = k\varphi(\alpha) + l\varphi(\beta),\, \alpha,\beta \in V,\, k,l \in \mathbb{K}$;
    \item 若 $\varphi$ 是同构,则 $\varphi$ 的逆映射 $\varphi^{-1}$ 也是线性映射,从而是 $U \to V$ 的同构。
  \end{enumerate}
\end{proposition}



%——————————————————————————————————%

\section{线性映射的运算}

\begin{definition}
  设 $\varphi,\psi$ 是 $\mathbb{K}$ 上线性空间 $V \to U$ 的线性映射。定义 $\varphi + \psi$ 为 $V \to U$ 的映射:
  \[(\varphi + \psi)(\alpha) = \varphi(\alpha) + \psi(\alpha),\, \alpha \in  V\]
  若 $k \in \mathbb{K}$,定义 $k\varphi$ 为 $V \to U$ 的映射:
  \[(k\varphi)(\alpha) = k\varphi(\alpha),\, \alpha \in V\]
\end{definition}

\begin{proposition}
  上面定义的映射 $\varphi + \psi$ 和 $k\varphi$ 都是线性映射。
\end{proposition}

\begin{proposition}
  设 $\mathcal{L}(V,U)$ 是 $V \to U$ 的线性映射的全体,则在线性映射 $\varphi + \psi$ 和 $k\varphi$ 下,$\mathcal{L}(V,U)$ 是 $\mathbb{K}$ 上的线性空间。

  特别地,$V \to \mathbb{K}$ 的所有线性函数全体构成一个线性空间,称为 $V$ 的共轭空间,记为 $V^*$。当 $V$ 是有限维空间时,$V^*$ 称为 $V$ 的对偶空间。
\end{proposition}

\begin{remark}
  若 $U = \mathbb{K}$,即把 $\mathbb{K}$ 看成 $\mathbb{K}$ 上的一维空间,则 $V \to \mathbb{K}$ 的线性映射称为线性函数。
\end{remark}

\begin{remark}
  若 $V = U$,则 $\mathcal{L}(V,V)$ 是 $V$ 上线性变换的全体构成的线性空间,简记为 $\mathcal{L}(V)$。$\mathcal{L}(V)$ 上还有乘法运算,即映射的复合。
\end{remark}

\begin{definition}
  设 $A$ 是 $\mathbb{K}$ 上的线性空间。若在 $A$ 上定义了乘法 $\cdot$(可省略),使得对 $\forall a,b,c \in A$,及 $k \in \mathbb{K}$,满足
  \begin{enumerate}
    \item $a(bc) = (ab)c$;
    \item $\exists e \in A,\, ea = ae = a$;
    \item $a(b + c) = ab + ac,\, (b + c)a = ba + ca$;
    \item $(ka)b = k(ab) = a(kb)$。
  \end{enumerate}
  则称 $A$ 是 $\mathbb{K}$ 上的代数,$e$ 称为 $A$ 的幺元。
\end{definition}

\begin{theorem}
  设 $V$ 是 $\mathbb{K}$ 上的线性空间,则 $\mathcal{L}(V)$ 是 $\mathbb{K}$ 上的代数。
\end{theorem}

\begin{proposition}
  在 $\mathcal{L}(V)$ 中,定义线性变换 $\varphi$ 的 $n$ 次幂为 $n$ 个 $\varphi$ 的复合,则
  \[\varphi^n \circ \varphi^m = \varphi^{n + m},\, (\varphi^n)^m = \varphi^{nm}\]
  若 $\varphi$ 是双射,即 $\varphi$ 是 $V$ 上的自同构,则 $\varphi^{-1}$ 也是 $V$ 上的线性变换,而且也是自同构,称 $\varphi^{-1}$ 是 $\varphi$ 的逆变换。定义
  \[\varphi^{-n} = (\varphi^{-1})^n\]
  则
  \[\varphi^{-n} = (\varphi^n)^{-1}\]
  定义
  \[\varphi^0 = I_V\]
  $\varphi$ 的负数次幂仅对自同构(又称可逆变换或非异变换)有意义。
\end{proposition}

\begin{proposition}
  若 $\varphi$ 和 $\psi$ 都是可逆线性变换,则 $\varphi \circ \psi$ 也是可逆线性变换,且
  \[(\varphi \circ \psi)^{-1} = \psi^{-1} \circ \varphi^{-1}\]
  对任意 $k \ne 0$,若 $\varphi$ 可逆,则 $k\varphi$ 也可逆,且
  \[(k\varphi)^{-1} = k^{-1}\varphi^{-1}\]
\end{proposition}


%——————————————————————————————————%

\section{线性映射与矩阵}

\begin{lemma}
  设 $\mathbb{K}$ 上的线性空间 $V$ 和 $U$,$\{e_1,e_2, \ldots ,e_n\}$ 是 $V$ 的一组基,则
  \begin{enumerate}
    \item 若有从 $V$ 到 $U$ 的映射 $\varphi$ 和 $\psi$ 满足 $\psi(e_i) = \varphi(e_i)(i = 1,2, \ldots ,n)$,则 $\psi = \varphi$;
    \item 给定 $U$ 中 $n$ 个向量 $\beta_1,\beta_2, \ldots ,\beta_n$,有且只有一个从 $V$ 到 $U$ 的线性映射 $\varphi$ 满足 $\varphi(e_i) = \beta_i(i = 1,2,\cdots,n')$。
  \end{enumerate}
\end{lemma}

\begin{definition}{线性映射在两个基下的表示矩阵}
  
\end{definition}


%——————————————————————————————————%

\section{线性映射的核与像}






%——————————————————————————————————%

\section{不变子空间}






%——————————————————————————————————%
