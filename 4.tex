\chapter{线性映射}

%——————————————————————————————————%

\section{线性映射的概念}

\begin{definition}{线性映射}
    设数域 $\mathbb{K}$ 上的线性空间 $V,U$。令映射 $\varphi : V \to U$。若
    \begin{enumerate}
        \item $\varphi(\alpha + \beta) = \varphi(\alpha) + \varphi(\beta),\, \alpha,\beta \in V$;
        \item $\varphi(k\alpha) = k\varphi(\alpha),\, k \in \mathbb{K},\alpha \in V$。
    \end{enumerate}
    则称 $\varphi$ 是 $V \to U$ 的线性映射。

    若 $\varphi V \to V$,则称 $\varphi$ 为线性变换。

    若映射 $\varphi$ 是单的,则称 $\varphi$ 为单线性映射。

    若映射 $\varphi$ 是满的,则称 $\varphi$ 为满线性映射。

    若映射 $\varphi$ 是双射,则称 $\varphi$ 为线性同构,简称同构。若 $V = U$,则 $V$ 自身上的同构称为自同构。
\end{definition}

\hfill

\begin{example}
    零映射:$\forall \alpha \in V, \varphi(\alpha) = \mathbf{0}$。记为 $\mathbf{0}$。
\end{example}

\hfill

\begin{example}
    恒等变换:恒等映射 $\mathbf{1}_V$ 是 $V$ 上的线性变换,记为 $I_V$,简记为 $I$。
\end{example}

\hfill

\begin{example}
    设数域 $\mathbb{K}$ 上的列向量空间 $V = \mathbb{K}_n, U = \mathbb{K}_m$ 和 $m \times n$ 矩阵 $A$,映射 $\varphi : V \to U$ 定义为
    \[\varphi(\alpha) = A\alpha\]

    $\varphi$ 由矩阵乘法定义,可得 $\varphi$ 是线性映射。
\end{example}

\hfill

\begin{proposition}
    设线性映射 $\varphi : V \to U$,则
    \begin{enumerate}
        \item $\varphi(0) = 0$;
        \item $\varphi(k\alpha + l\beta) = k\varphi(\alpha) + l\varphi(\beta),\, \alpha,\beta \in V,\, k,l \in \mathbb{K}$;
        \item 若 $\varphi$ 是同构,则 $\varphi$ 的逆映射 $\varphi^{-1}$ 也是线性映射,从而是 $U \to V$ 的同构。
    \end{enumerate}
\end{proposition}



%——————————————————————————————————%

\section{线性映射的运算}






%——————————————————————————————————%

\section{线性映射与矩阵}






%——————————————————————————————————%

\section{线性映射的核与像}






%——————————————————————————————————%

\section{不变子空间}






%——————————————————————————————————%

\section{}






%——————————————————————————————————%

\section{}






%——————————————————————————————————%

\section{}






%——————————————————————————————————%

\section{}






%——————————————————————————————————%

\section{}






%——————————————————————————————————%

\section{}






%——————————————————————————————————%

\section{}






