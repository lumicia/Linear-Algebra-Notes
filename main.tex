\documentclass[cn,10pt,math=newtx,citestyle=gb7714-2015,bibstyle=gb7714-2015]{elegantbook}

\usepackage{CJKfntef}
\usepackage{amssymb}
\usepackage{array}
\usepackage[utf8]{inputenc}
%\usepackage{caption}
\usepackage{chemformula}
\usepackage{svg}
\usepackage{extarrows}
%\usepackage{0-figure/extrarelation}
\usepackage{tabularray}
\usepackage{hyperref}
\hypersetup{
colorlinks=true,
linkcolor=blue,
anchorcolor=blue,
citecolor=blue}
\usepackage{bm}

\DeclareMathOperator{\im}{im}

\cover{figure/cover.jpg}

\definecolor{customcolor}{RGB}{159,168,218}
\colorlet{coverlinecolor}{customcolor}

\title{线性代数}
\author{lumicia}
\date{\today}
\version{0.1}
\extrainfo{合抱之木,生于毫末;九层之台,起于累土;千里之行,始于足下。——老子}

\begin{document}
\maketitle
\frontmatter

\newpage

\tableofcontents
\mainmatter

% \newpage

% \input{1}
% \newpage

% \input{2}
% \newpage

\chapter{线性空间}

%——————————————————————————————————%

\section{}






%——————————————————————————————————%

\section{}






%——————————————————————————————————%

\section{}






%——————————————————————————————————%

\section{}






%——————————————————————————————————%

\section{}






%——————————————————————————————————%

\section{}






%——————————————————————————————————%

\section{}






%——————————————————————————————————%

\section{}






%——————————————————————————————————%

\section{}






%——————————————————————————————————%

\section{}






%——————————————————————————————————%

\section{}






%——————————————————————————————————%

\section{}







\newpage

\chapter{线性映射}

%——————————————————————————————————%

\section{线性映射的概念}

\begin{definition}{线性映射}
    设数域 $\mathbb{K}$ 上的线性空间 $V,U$。令映射 $\varphi : V \to U$。若
    \begin{enumerate}
        \item $\varphi(\alpha + \beta) = \varphi(\alpha) + \varphi(\beta),\, \alpha,\beta \in V$;
        \item $\varphi(k\alpha) = k\varphi(\alpha),\, k \in \mathbb{K},\alpha \in V$。
    \end{enumerate}
    则称 $\varphi$ 是 $V \to U$ 的线性映射。

    若 $\varphi V \to V$,则称 $\varphi$ 为线性变换。

    若映射 $\varphi$ 是单的,则称 $\varphi$ 为单线性映射。

    若映射 $\varphi$ 是满的,则称 $\varphi$ 为满线性映射。

    若映射 $\varphi$ 是双射,则称 $\varphi$ 为线性同构,简称同构。若 $V = U$,则 $V$ 自身上的同构称为自同构。
\end{definition}

\hfill

\begin{example}
    零映射:$\forall \alpha \in V, \varphi(\alpha) = \mathbf{0}$。记为 $\mathbf{0}$。
\end{example}

\hfill

\begin{example}
    恒等变换:恒等映射 $\mathbf{1}_V$ 是 $V$ 上的线性变换,记为 $I_V$,简记为 $I$。
\end{example}

\hfill

\begin{example}
    设数域 $\mathbb{K}$ 上的列向量空间 $V = \mathbb{K}_n, U = \mathbb{K}_m$ 和 $m \times n$ 矩阵 $A$,映射 $\varphi : V \to U$ 定义为
    \[\varphi(\alpha) = A\alpha\]

    $\varphi$ 由矩阵乘法定义,可得 $\varphi$ 是线性映射。
\end{example}

\hfill

\begin{proposition}
    设线性映射 $\varphi : V \to U$,则
    \begin{enumerate}
        \item $\varphi(0) = 0$;
        \item $\varphi(k\alpha + l\beta) = k\varphi(\alpha) + l\varphi(\beta),\, \alpha,\beta \in V,\, k,l \in \mathbb{K}$;
        \item 若 $\varphi$ 是同构,则 $\varphi$ 的逆映射 $\varphi^{-1}$ 也是线性映射,从而是 $U \to V$ 的同构。
    \end{enumerate}
\end{proposition}



%——————————————————————————————————%

\section{线性映射的运算}

\begin{definition}
    设 $\varphi,\psi$ 是 $\mathbb{K}$ 上线性空间 $V \to U$ 的线性映射。定义 $\varphi + \psi$ 为 $V \to U$ 的映射:
    \[(\varphi + \psi)(\alpha) = \varphi(\alpha) + \psi(\alpha),\, \alpha \in  V\]
    若 $k \in \mathbb{K}$,定义 $k\varphi$ 为 $V \to U$ 的映射:
    \[(k\varphi)(\alpha) = k\varphi(\alpha),\, \alpha \in V\]
\end{definition}

\begin{proposition}
    上面定义的映射 $\varphi + \psi$ 和 $k\varphi$ 都是线性映射。
\end{proposition}

\begin{proposition}
    设 $\mathcal{L}(V,U)$ 是 $V \to U$ 的线性映射的全体,则在线性映射 $\varphi + \psi$ 和 $k\varphi$ 下,$\mathcal{L}(V,U)$ 是 $\mathbb{K}$ 上的线性空间。

    特别地,$V \to \mathbb{K}$ 的所有线性函数全体构成一个线性空间,称为 $V$ 的共轭空间,记为 $V^*$。当 $V$ 是有限维空间时,$V^*$ 称为 $V$ 的对偶空间。
\end{proposition}

\begin{remark}
    若 $U = \mathbb{K}$,即把 $\mathbb{K}$ 看成 $\mathbb{K}$ 上的一维空间,则 $V \to \mathbb{K}$ 的线性映射称为线性函数。
\end{remark}

\begin{remark}
  若 $V = U$,则 $\mathcal{L}(V,V)$ 是 $V$ 上线性变换的全体构成的线性空间,简记为 $\mathcal{L}(V)$。$\mathcal{L}(V)$ 上还有乘法运算,即映射的复合。
\end{remark}

\begin{definition}
  设 $A$ 是 $\mathbb{K}$ 上的线性空间。若在 $A$ 上定义了乘法 $\cdot$(可省略),使得对 $\forall a,b,c \in A$,及 $k \in \mathbb{K}$,满足
  \begin{enumerate}
    \item $a(bc) = (ab)c$;
    \item $\exists e \in A,\, ea = ae = a$;
    \item $a(b + c) = ab + ac,\, (b + c)a = ba + ca$;
    \item $(ka)b = k(ab) = a(kb)$。
  \end{enumerate}
  则称 $A$ 是 $\mathbb{K}$ 上的代数,$e$ 称为 $A$ 的幺元。
\end{definition}

\begin{theorem}
  设 $V$ 是 $\mathbb{K}$ 上的线性空间,则 $\mathcal{L}(V)$ 是 $\mathbb{K}$ 上的代数。
\end{theorem}

\begin{proposition}
  在 $\mathcal{L}(V)$ 中,定义线性变换 $\varphi$ 的 $n$ 次幂为 $n$ 个 $\varphi$ 的复合,则
  \[\varphi^n \circ \varphi^m = \varphi^{n + m},\, (\varphi^n)^m = \varphi^{nm}\]
  若 $\varphi$ 是双射,即 $\varphi$ 是 $V$ 上的自同构,则 $\varphi^{-1}$ 也是 $V$ 上的线性变换,而且也是自同构,称 $\varphi^{-1}$ 是 $\varphi$ 的逆变换。定义
  \[\varphi^{-n} = (\varphi^{-1})^n\]
  则
  \[\varphi^{-n} = (\varphi^n)^{-1}\]
  定义
  \[\varphi^0 = I_V\]
  $\varphi$ 的负数次幂仅对自同构(又称可逆变换或非异变换)有意义。
\end{proposition}

\begin{proposition}
  若 $\varphi$ 和 $\psi$ 都是可逆线性变换,则 $\varphi \circ \psi$ 也是可逆线性变换,且
  \[(\varphi \circ \psi)^{-1} = \psi^{-1} \circ \varphi^{-1}\]
  对任意 $k \ne 0$,若 $\varphi$ 可逆,则 $k\varphi$ 也可逆,且
  \[(k\varphi)^{-1} = k^{-1}\varphi^{-1}\]
\end{proposition}


%——————————————————————————————————%

\section{线性映射与矩阵}






%——————————————————————————————————%

\section{线性映射的核与像}






%——————————————————————————————————%

\section{不变子空间}






%——————————————————————————————————%

\section{}






%——————————————————————————————————%

\section{}






%——————————————————————————————————%

\section{}






%——————————————————————————————————%

\section{}






%——————————————————————————————————%

\section{}






%——————————————————————————————————%

\section{}






%——————————————————————————————————%

\section{}







\newpage

%\input{5}
%\newpage

\chapter{特征值}

%——————————————————————————————————%

\section{特征值和特征向量}

\begin{definition}
    设 $\varphi$ 是数域 $\mathbb{K}$ 上线性空间 $V$ 上的线性变换。若 $\lambda_0 \in \mathbb{K}, \bm{x} \in V$ 且 $\bm{x} \neq 0$,使
    \[
        \varphi\bm{x} = \lambda_{0}\bm{x}
    \]
    则称 $\lambda_0$ 是线性变换 $\varphi$ 的特征值,向量 $\bm{x}$ 称为 $\varphi$ 关特征值 $\lambda_0$ 的特征向量。
\end{definition}

\begin{remark}
    $\varphi$ 关于特征值 $\lambda_0$ 的全体特征向量加上零向量构成 $V$ 的子空间,记为 $V_{\lambda_{0}}$,称为 $\varphi$ 的关于特征值 $\lambda_0$ 的特征子空间。显然 $V_{\lambda_{0}}$ 是不变子空间。
\end{remark}

\begin{definition}
    设 $\bm{A}$ 上数域 $\mathbb{K}$ 上的 $n$ 阶方阵。若 $\exists \lambda_0 \in \mathbb{K}$ 及 $n$ 维非零列向量 $\bm{\alpha}$ 使得
    \[
        \bm{A\alpha} = \lambda_{0}\bm{\alpha}
    \]
    成立,则称 $\lambda_0$ 为矩阵 $\bm{A}$ 的特征值,$\bm{\alpha}$ 为 $\bm{A}$ 关于特征值 $\lambda_0$ 的特征向量。齐次线性方程组
    \[
        (\lambda_{0}\bm{I}_n - \bm{A})\bm{x} = 0
    \]
    的解空间 $V_{\lambda_{0}}$ 称为 $\bm{A}$ 关于特征值 $\lambda_0$ 的特征子空间。
\end{definition}

\begin{definition}
    设 $\bm{A}$ 是 $n$ 阶方阵,称 $\vert \lambda \bm{I}_n - \bm{A} \vert$ 为 $\bm{A}$ 的特征多项式。
\end{definition}

\begin{theorem}
    若 $\bm{B}$ 与 $\bm{A}$ 相似,则 $\bm{B}$ 与 $\bm{A}$ 有相同的特征多项式,从而具有相同的特征值(计重数)。
\end{theorem}

\begin{definition}
    设 $\varphi$ 是线性空间 $V$ 上的线性变换,$\varphi$ 在 $V$ 上的某组基下的表示矩阵为 $\bm{A}$,由上面的定理可知 $\left\vert \lambda \bm{I}_n - \bm{A} \right\vert$ 与基或表示矩阵的选取无关,称 $\left\vert \lambda \bm{I}_n - \bm{A} \right\vert$ 为 $\varphi$ 的特征多项式,记为 $\left\vert \lambda \bm{I}_V - \varphi \right\vert$。
\end{definition}

\begin{theorem}
    任一复方阵必定相似于一上三角阵。
\end{theorem}

\begin{proposition}
    设 $n$ 阶矩阵 $\bm{A}$ 的全体特征值为 $\lambda_1, \lambda_2, \ldots, \lambda_n$,多项式 $f(x) = a_{m}x^{m} + a_{m - 1}x^{m - 1} + \cdots +a_{1}x + a_{0}$。记
    \[f(\bm{A}) = a_{m}\bm{A}^{m} + a_{m - 1}\bm{A}^{m - 1} + \cdots +a_{1}\bm{A} + a_{0}\bm{I}_n\]
    则 $f(\bm{A})$ 的全部特征值为 $f(\lambda_1), f(\lambda_2), \ldots, f(\lambda_n)$。
\end{proposition}

\begin{proposition}
    设 $n$ 阶矩阵 $\bm{A}$ 适合多项式 $g(x)$,即 $g(\bm{A}) = \bm{O}$,则 $\bm{A}$ 的任一特征值 $\lambda_0$ 也必定适合 $g(x)$,即 $g(\lambda_0) = 0$。
\end{proposition}

\begin{proposition}
    设 $n$ 阶矩阵 $\bm{A}$ 是可逆阵,且 $\bm{A}$ 的全部特征值为 $\lambda_1, \lambda_2, \ldots , \lambda_n$,则 $\bm{A}^{-1}$ 的全部特征值为 $\lambda_{1}^{-1}, \lambda_{2}^{-1}, \ldots ,\lambda_{n}^{-1}$。
\end{proposition}





%——————————————————————————————————%

\section{对角化}

\begin{definition}
    若 $n$ 维线性空间 $V$ 上的线性变换 $\varphi$ 在某组基下 $\{e_1, e_2, \ldots , e_n\}$ 下的表示矩阵为对角阵
    \[
        \begin{pmatrix}
            \lambda_1 & \         & \      & \         \\
            \         & \lambda_2 & \      & \         \\
            \         & \         & \ddots & \         \\
            \         & \         & \      & \lambda_n
        \end{pmatrix}
    \]
    则称 $\varphi$ 为可对角化线性变换。此时 $\varphi(e_{i}) = \lambda_{i}e_{i}$,即 $e_1, e_2, \ldots , e_n$ 是 $\varphi$ 的特征向量,于是 $\varphi$ 有 $n$ 个线性无关的特征向量。
\end{definition}

\begin{theorem}{$\varphi$ 可对角化的第一个充分必要条件}
    设 $\varphi$ 是 $n$ 维线性空间 $V$ 上的线性变换,则 $\varphi$ 可对角化的充分必要条件是 $\varphi$ 有 $n$ 个线性无关的特征向量。
\end{theorem}

\begin{definition}
    设 $\bm{A}$ 是 $n$ 阶矩阵,若 $\bm{A}$ 相似于对角阵,即存在可逆阵 $\bm{P}$ 使得 $\bm{P}^{-1}\bm{AP}$ 为对角阵,则称 $\bm{A}$ 是可对角化矩阵。
\end{definition}

\begin{theorem}{$\bm{A}$ 可对角化的第一个充分必要条件}
    设 $\bm{A}$ 是 $n$ 阶矩阵,则 $\bm{A}$ 可对角化的充分必要条件是 $\bm{A}$ 有 $n$ 个线性无关的特征向量。
\end{theorem}

\begin{theorem}
    若 $\lambda_1, \lambda_2, \ldots , \lambda_n$ 为 $n$ 为线性空间 $V$ 上的线性变换 $\varphi$ 的不同的特征值,则
    \[
        V_1 + V_2 + \cdots + V_k = V_1 \oplus V_2 \oplus \cdots \oplus V_k
    \]
\end{theorem}

\begin{corollary}
    线性变换 $\varphi$ 属于不同特征值的特征向量必定线性无关。
\end{corollary}

\begin{corollary}
    若 $n$ 维线性空间 $V$ 上的线性变换 $\varphi$ 有 $n$ 个不同的特征值,则 $\varphi$ 必定可对角化。
\end{corollary}

\begin{theorem}{$\varphi$ 可对角化的第二个充分必要条件}
    设 $\varphi$ 是 $n$ 维线性空间 $V$ 上的线性变换,$\lambda_1, \lambda_2, \ldots , \lambda_k$ 是 $\varphi$ 的全部不同特征值,$V_i(i = 1, 2, \cdots, k)$ 是特征值 $\lambda_i$ 的特征子空间,则 $\varphi$ 可对角化的充分必要条件是
    \[
        V = V_1 \oplus V_2 \oplus \cdots \oplus V_k
    \]
\end{theorem}

\begin{theorem}{$\bm{A}$ 可对角化的第二个充分必要条件}
    设 $\bm{A}$ 是 $n$ 阶矩阵,$\lambda_1, \lambda_2, \ldots , \lambda_k$ 是 $\bm{A}$ 的全部不同特征值,$V_i(i = 1, 2, \cdots, k)$ 是特征值 $\lambda_i$ 的特征子空间,则 $\bm{A}$ 可对角化的充分必要条件是
    \[
        V = V_1 \oplus V_2 \oplus \cdots \oplus V_k
    \]
\end{theorem}

\begin{definition}
    设 $\varphi$ 是 $n$ 维线性空间 $V$ 上的线性变换,$\lambda_0$ 是 $\varphi$ 的特征值,$V_0$ 是属于 $\lambda_0$ 的特征子空间,称 $\dim V_0$ 为 $\lambda_0$ 的度数或几何重数。 $\lambda_0$ 作为 $\varphi$ 的特征多项式根的重数称为 $\lambda_0$ 的重数或代数重数。
\end{definition}

\begin{lemma}
    设 $\varphi$ 是 $n$ 维线性空间 $V$ 上的线性变换,$\lambda_0$ 是 $\varphi$ 的特征值,则 $\lambda_0$ 的度数总是小于等于 $\lambda_0$ 的度数。
\end{lemma}

\begin{definition}
    设 $\varphi$ 是 $n$ 维线性空间 $V$ 上的线性变换。若 $\varphi$ 的任一特征值的度数等于重数,则称 $\varphi$ 有完全的特征向量系。
\end{definition}

\begin{theorem}{$\varphi$ 可对角化的第三个充分必要条件}
    设 $\varphi$ 是 $n$ 维线性空间 $V$ 上的线性变换,则 $\varphi$ 可对角化的充分必要条件是 $\varphi$ 有完全的特征向量系。
\end{theorem}

\begin{theorem}{$\bm{A}$ 可对角化的第三个充分必要条件}
    设 $\bm{A}$ 是 $n$ 阶矩阵,则 $\bm{A}$ 可对角化的充分必要条件是 $\varphi$ 有完全的特征向量系。
\end{theorem}



%——————————————————————————————————%

\section{极小多项式与 Cay-Hamilton 定理}

\begin{definition}{极小多项式}
    若 $n$ 阶矩阵 $\bm{A}$(或 $n$ 维线性空间 $V$ 上的线性变换 $\varphi$)适合非零首一多项式 $m(x)$,且 $m(x)$ 是 $\bm{A}$(或 $\varphi$)所适合的非零多项式中次数最小者,则称 $m(x)$ 是 $\bm{A}$(或 $\varphi$)的极小多项式或最小多项式。
\end{definition}

\begin{lemma}
    若 $f(x)$ 是 $\bm{A}$ 适合的多项式,则 $\bm{A}$ 的极小多项式 $m(x)$ 整除 $f(x)$。
\end{lemma}

\begin{proposition}
    任一 $n$ 阶矩阵的极小多项式必定唯一。
\end{proposition}

\begin{proposition}
    相似的矩阵具有相同的极小多项式。
\end{proposition}

\begin{proposition}
    设 $\bm{A}$ 是分块对角阵
    \[
        \bm{A} = \begin{pmatrix}
            \bm{A}_{1} & \      & \      & \      \\
            \      & \bm{A}_{2} & \      & \      \\
            \      & \      & \ddots & \      \\
            \      & \      & \      & \bm{A}_{n}
        \end{pmatrix}
    \]
    其中 $\bm{A}_{i}$ 都是方阵,则 $\bm{A}$ 的极小多项式等于各个 $\bm{A}_{i}$ 的极小多项式的最小公倍式。
\end{proposition}

\begin{lemma}
    设 $m(x)$ 是 $n$ 阶矩阵 $\bm{A}$ 的极小多项式,$\lambda_0$ 是 $\bm{A}$ 的特征值,则
    \[
        (x - \lambda_0) | m(x)
    \]
\end{lemma}

\begin{theorem}{Cayley-Hamilton 定理}
    设 $\bm{A}$ 是数域 $\mathbb{K}$ 上的 $n$ 阶矩阵,$f(x)$ 是 $\bm{A}$ 的特征多项式,则 $f(\bm{A}) = \bm{O}$。
\end{theorem}

\begin{corollary}
    $n$ 阶矩阵 $\bm{A}$ 的极小多项式是其特征多项式的因式。特别地,$\bm{A}$ 的极小多项式的次数不超过 $n$。
\end{corollary}

\begin{corollary}
    $n$ 阶矩阵 $\bm{a}$ 的极小多项式和特征多项式具有相同的根(不计重数)。
\end{corollary}

\begin{corollary}
    设 $\varphi$ 是 $n$ 维线性空间 $V$ 上的线性变换,$f(x)$ 是 $\varphi$ 的特征多项式,则 $f(\varphi) = \bm{0}$。
\end{corollary}
\begin{remark}
    该推论也被称为 Cayley-Hamilton 定理。
\end{remark}





%——————————————————————————————————%

\section{特征值的估计}

\begin{theorem}{Gerschgorin 圆盘第一定理}
    设 $\bm{A} = (a_{ij})_{n \times n}$ 是 $n$ 阶复矩阵,则 $\bm{A}$ 的特征值在复平面上的下列圆盘(又称为戈氏圆盘)中:
    \[
        \vert z - a_{ii} \vert \leqslant R_i,\quad i = 1, 2, \cdots, n
    \]
\end{theorem}

\begin{theorem}
    设
    \[
        f(x) = x^{n} + a_{1}x^{n - 1} + \cdots + a_{n - 1}x + a_{n}
    \]
    是 $n$ 次首一复系数多项式,则 $f(x)$ 的 $n$ 个根 $\lambda_1, \lambda_2, \ldots, \lambda_n$ 作为整体是 $a_1, a_2, \ldots, a_n$ 的连续函数。
\end{theorem}

\begin{theorem}
  设 $D$是实轴上的区间,$\bm{A}: D \to M_n(\mathbb{C})$ 是取值为复矩阵的连续函数,则 $\bm{A}(t)$ 的 $n$ 个特征值 $\lambda_1(t), \lambda_2(t), \ldots, \lambda_n(t)$ 都是关于 $t$ 的连续函数。
\end{theorem}

\begin{theorem}
  设矩阵 $\bm{A} = (a_{ij})_{n \times n}$ 的 $n$ 个戈氏圆盘分成若干个连通区域,若其中一个连通区域含有 $k$ 个戈氏圆盘,则有且只有 $k$ 个特征值落在这个连通区域内(若两个戈氏圆盘重合,需计重数;又若特征值为重根,也计重数)。
\end{theorem}


\newpage

%\chapter{相似标准型}

%——————————————————————————————————%

\section{多项式矩阵}

\begin{definition}
    若矩阵中每个元素都是以 $\lambda$ 为未定元的数域 $\mathbb{K}$ 上的多项式,则称该矩阵为多项式矩阵或 $\lambda$ -矩阵,记为
    \[
        \bm{A}(\lambda) = \begin{pmatrix}
            a_{11}(\lambda) & a_{12}(\lambda) & \cdots & a_{1n}(\lambda) \\
            a_{21}(\lambda) & a_{22}(\lambda) & \cdots & a_{2n}(\lambda) \\
            \vdots          & \vdots          & \      & \vdots          \\
            a_{m1}(\lambda) & a_{m2}(\lambda) & \cdots & a_{mn}(\lambda) \\
        \end{pmatrix}
    \]
\end{definition}

\begin{definition}
    对 $\lambda$ -矩阵 $\bm{A}(\lambda)$ 施行的下列三种变换称为 $\lambda$ -矩阵的初等行变换:
    \begin{enumerate}
        \item 将 $\bm{A}(\lambda)$ 的两行对换;
        \item 将 $\bm{A}(\lambda)$ 的第 $i$ 行乘以 $\mathbb{K}$ 中的非零常数 $c$;
        \item 将 $\bm{A}(\lambda)$ 的第 $i$ 行乘以 $\mathbb{K}$ 中的多项式 $f(\lambda)$ 后加到第 $j$ 行上去。
    \end{enumerate}
\end{definition}

\begin{definition}
    若 $\bm{A}(\lambda)$ 和 $\bm{B}(\lambda)$ 是同阶 $\lambda$ -矩阵且 $\bm{A}(\lambda)$ 经过 $\lambda$ -矩阵的初等变换后可变为 $\bm{B}(\lambda)$,则称 $\bm{A}(\lambda)$ 与 $\bm{B}(\lambda)$ 相抵。
\end{definition}

\begin{remark}
    $\lambda$ -矩阵的相抵关系也是一种等价关系。
\end{remark}

\begin{definition}
    下列三种矩阵称为初等 $\lambda$ -矩阵:
    \begin{enumerate}
        \item 将 $n$ 阶单位阵的第 $i$ 行与第 $j$ 行对换,记为 $\bm{P}_{ij}$;
        \item 将 $n$ 阶单位阵的第 $i$ 行乘以非零常数 $c$,记为 $\bm{P}_{i}(c)$;
        \item 将 $n$ 阶单位阵的第 $i$ 行乘以多项式 $f(\lambda)$ 后加到第 $j$ 行上去得到的矩阵,记为 $\bm{T}_{ij}(f(\lambda))$。
    \end{enumerate}
\end{definition}

\begin{theorem}
    对 $\lambda$ -矩阵 $\bm{A}(\lambda)$ 施以第 $k(k = 1, 2, 3)$ 类初等行(列)变换等于用第 $k$ 类初等 $\lambda$ -矩阵左(右)乘以 $\bm{A}(\lambda)$。
\end{theorem}

\begin{definition}
    若 $\bm{A}(\lambda)$ 和 $\bm{B}(\lambda)$ 都是 $n$ 阶 $\lambda$ -矩阵,且
    \[
        \bm{A}(\lambda)\bm{B}(\lambda) = \bm{B}(gl)\bm{A}(\lambda) = \bm{I}_{n}
    \]
    则称 $\bm{B}(\lambda)$ 是 $\bm{A}(\lambda)$ 的逆 $\lambda$ -矩阵。这时称 $\bm{A}(\lambda)$ 为可逆 $\lambda$ -矩阵,在不引起混淆的情况下,有时简称为可逆阵。
\end{definition}

\begin{lemma}
    设 $\bm{M}(\lambda)$ 与 $\bm{N}(\lambda)$ 是两个 $n$ 阶 $\lambda$ - 矩阵且都不等于零。又设 $\bm{B}$ 为 $n$ 阶数字矩阵,则必定存在 $\lambda$ -矩阵 $\bm{Q}(\lambda)$ 及 $\bm{S}(\lambda)$ 和数字矩阵 $\bm{R}$ 及 $\bm{T}$,使得下式成立:
    \begin{align*}
        \bm{M}(\lambda) = (\lambda \bm{I} - \bm{B})\bm{Q}(\lambda) + \bm{R} \\
        \bm{N}(\lambda) = \bm{S}(\lambda)(\lambda \bm{I} - \bm{B}) + \bm{T}
    \end{align*}
\end{lemma}

\begin{theorem}
    设 $\bm{A}, \bm{B}$ 是数域 $\mathbb{K}$ 上的矩阵,则 $\bm{A}$ 与 $\bm{B}$ 相似的充分必要条件是 $\lambda$ -矩阵 $\lambda \bm{I} - \bm{A}$ 与 $\lambda \bm{I} - \bm{B}$ 相抵。
\end{theorem}



%——————————————————————————————————%

\section{矩阵的法式}

\begin{lemma}
    设 $\bm{A}(\lambda) = (a_{ij}(\lambda))_{m \times n}$ 是任一非零 $\lambda$ -矩阵,则 $\bm{A}(\lambda)$ 必相抵于这样的一个 $\lambda$ -矩阵 $\bm{B}(\lambda) = (b_{ij}(\lambda))_{m \times n}$,其中 $b_{11}(\lambda) \neq 0$ 且 $b_{11}(\lambda)$ 可整除 $\bm{B}(gl)$ 中的任一元素 $b_{ij}(\lambda)$。
\end{lemma}

\begin{theorem}
    设 $\bm{A}(\lambda)$ 是 $n$ 阶 $\lambda$ -矩阵,则 $\bm{A}(\lambda)$ 相抵于对角阵
    \[
        \operatorname{diag}\{ d_{1}(\lambda), d_{2}(\lambda), \ldots, d_{r}(\lambda); 0, \ldots, 0 \}
    \]
    其中 $d_{i}(\lambda)$ 是非零首一多项式且 $d_{i}(\lambda) \mid d_{i + 1}(\lambda)(i = 1, 2, \ldots, r - 1)$。
\end{theorem}

\begin{corollary}
    任一 $n$ 阶可逆 $\lambda$ -矩阵都可表示为有限个 $\lambda$ -矩阵之积。
\end{corollary}

\begin{corollary}
    设 $\bm{A}$ 是数域 $\mathbb{K}$ 上的 $n$ 阶矩阵,则 $\bm{A}$ 的特征矩阵 $\lambda \bm{I}_{n} - \bm{A}$ 必定相抵于
    \[
        \operatorname{diag}\{ 1, \ldots, 1, d_{1}(\lambda), \ldots, d_{m}(\lambda) \}
    \]
    其中 $d_{i}(\lambda) \mid d_{i + 1}(\lambda)(i = 1, 2, \ldots, m - 1)$。
\end{corollary}


\begin{definition}
    对角 $\lambda$ -矩阵
    \[
        \operatorname{diag}\{ d_{1}(\lambda), d_{2}(\lambda), \ldots, d_{r}(\lambda); 0, \ldots, 0 \}
    \]
    称为 $\bm{A}(\lambda)$ 的法式或相抵标准型。
\end{definition}


%——————————————————————————————————%

\section{不变因子}

\begin{definition}
    设 $\bm{A}(\lambda)$ 是 $n$ 阶 $\lambda$ -矩阵,$k \in \mathbb{N}^*$ 且 $k \leqslant n$。如果 $\bm{A}(\lambda)$ 有一个 $k$ 阶子式不为零,则定义 $\bm{A}(\lambda)$ 的 $k$ 阶行列式因子 $D_{k}(\lambda)$ 为 $\bm{A}(\lambda)$ 的所有 $k$ 阶子式的最大公因式(首一多项式)。如果 $\bm{A}(\lambda)$ 的所有 $k$ 阶子式都等于零,则定义 $\bm{A}(\lambda)$ 的 $k$ 阶行列式因子 $D_{k}(\lambda)$ 为零。
\end{definition}

\begin{lemma}
    设 $D_{1}(\lambda), D_{2}(\lambda), \ldots, D_{r}(\lambda)$ 是 $\bm{A}(\lambda)$ 的非零行列式因子,则
    \[
        D_{i}(\lambda) \mid D_{i + 1}(\lambda),\quad i = 1, 2, \ldots, r - 1
    \]
\end{lemma}

\begin{definition}
    设 $D_{1}(\lambda), D_{2}(\lambda), \ldots, D_{r}(\lambda)$ 是 $\bm{A}(\lambda)$ 的非零行列式因子,则 $g_{1}(\lambda) = D_{1}(\lambda), g_{2}(\lambda) = \dfrac{D_{2}(\lambda)}{D_{1}(\lambda)}, \ldots, g_{r}(\lambda) = \dfrac{D_{r}(\lambda)}{D_{r - 1}(\lambda)}$ 称为 $\bm{A}(\lambda)$ 的不变因子。
\end{definition}

\begin{theorem}
    相抵的 $\lambda$ -矩阵有相同的行列式因子,从而有相同的不变因子。
\end{theorem}

\begin{corollary}
    设 $n$ 阶 $\lambda$ -矩阵 $\bm{A}(\lambda)$ 的法式为
    \[
        \varLambda = \operatorname{diag}\{ d_{1}(\lambda), d_{2}(\lambda), \ldots, d_{r}(\lambda); 0, \ldots, 0 \}
    \]
    其中 $d_{i}(\lambda)$ 是非零首一多项式且 $d_{i}(\lambda) \mid d_{i + 1}(\lambda)(i = 1, 2, \ldots, r - 1)$,则 $\bm{A}(\lambda)$ 的不变因子为
    \[
        d_{1}(\lambda), d_{2}(\lambda), \ldots, d_{r}(\lambda)
    \]
    特别地,法式和不变因子之间相互唯一确定。
\end{corollary}

\begin{corollary}
    设 $\bm{A}(\lambda), \bm{B}(\lambda)$ 为 $n$ 阶 $\lambda$ -矩阵,则 $\bm{A}(\lambda)$ 与 $\bm{B}(\lambda)$ 相抵当且仅当它们有相同的法式。
\end{corollary}

\begin{corollary}
    $n$ 阶 $\lambda$ -矩阵 $\bm{A}(\lambda)$ 的法式与初等变换的选取无关。
\end{corollary}

\begin{theorem}
    数域 $\mathbb{K}$ 上的 $n$ 阶矩阵 $\bm{A}$ 与 $\bm{B}$ 相似的充分必要条件是它们的特征矩阵 $\lambda \bm{I} - \bm{A}$ 与 $\lambda \bm{I} - \bm{B}$ 有相同的行列式因子或不变因子。
\end{theorem}

\begin{corollary}
    设 $\mathbb{F} \subseteq \mathbb{K}$ 是两个数域,$\bm{A}, \bm{B}$ 是 $\mathbb{F}$ 上的两个矩阵,则 $\bm{A}$ 与 $\bm{B}$ 在 $\mathbb{F}$ 上相似的充分必要条件是它们在 $\mathbb{K}$ 上相似。
\end{corollary}


%——————————————————————————————————%

\section{有理标准型}

\begin{lemma}\label{lem:7.4.1}
    设 $r$ 阶矩阵
    \[
        \bm{F} = \begin{pmatrix}
            0      & 1          & 0          & \cdots & 0      \\
            0      & 0          & 1          & \cdots & 0      \\
            \vdots & \vdots     & \vdots     & \      & \vdots \\
            0      & 0          & 0          & \cdots & 1      \\
            -a_{r} & -a_{r - 1} & -a_{r - 2} & \cdots & -a_{1}
        \end{pmatrix}
    \]
    则
    \begin{enumerate}
        \item $\bm{F}$ 的行列式因子为
              \begin{equation}\tag{1}
                  1, \ldots, 1, f(\lambda)
              \end{equation}
              其中共有 $r - 1$ 个 $1$,$f(\lambda) = \lambda^r + a_{1}\lambda^{r - 1} + \cdots + a_{r}$,$\bm{F}$ 的不变因子也由 $(1)$ 式给出。
        \item $\bm{F}$ 的极小多项式等于 $f(\lambda)$。
    \end{enumerate}
\end{lemma}

\begin{lemma}
    设 $\lambda$ -矩阵 $\bm{A}(\lambda)$ 相抵于对角 $\lambda$ -矩阵
    \[
        \operatorname{diag}\{ d_{1}(\lambda), d_{2}(\lambda), \ldots, d_{n}(\lambda) \}
    \]
    $\lambda$ -矩阵 $\bm{B}(\lambda)$ 相抵于对角 $\lambda$ -矩阵
    \[
        \operatorname{diag}\{ d_{1}^{'}(\lambda), d_{2}^{'}(\lambda), \ldots, d_{n}^{'}(\lambda) \}
    \]
    且 $d_{1}^{'}(\lambda), d_{2}^{'}(\lambda), \ldots, d_{n}^{'}(\lambda)$ 是 $d_{1}(\lambda), d_{2}(\lambda), \ldots, d_{n}(\lambda)$ 的置换(即若不计次序,这两组多项式完全相同)。
\end{lemma}

\begin{theorem}
    设 $\bm{A}$ 是数域 $\mathbb{K}$ 上的 $n$ 阶方阵,$\bm{A}$ 的不变因子组为
    \[
        1, \ldots, 1, d_{1}(\lambda), \ldots, d_{k}(\lambda)
    \]
    其中 $\deg d_{i}(\lambda) = m_{i} \geqslant 1$,则 $\bm{A}$ 相似于下列分块对角阵:
    \begin{equation}
        \bm{F} = \begin{pmatrix}
            \bm{F}_{1} & \          & \                   \\
            \          & \bm{F}_{2} & \      & \          \\
            \          & \          & \ddots & \          \\
            \          & \          & \      & \bm{F}_{k}
        \end{pmatrix}\label{equ:7.4.4}
    \end{equation}
    其中 $\bm{F}_i$ 的阶等于 $m_i$,且 $\bm{F}_i$ 是形如 $\ref{lem:7.4.1}$ 中的 $\bm{F}$ 矩阵, $\bm{F}_i$ 的最后一行由 $d_{i}(\lambda)$ 的系数(除首项系数之外)的负值组成。
\end{theorem}

\begin{definition}
    我们将分块对角阵
    \begin{equation*}
        \bm{F} = \begin{pmatrix}
            \bm{F}_{1} & \          & \                   \\
            \          & \bm{F}_{2} & \      & \          \\
            \          & \          & \ddots & \          \\
            \          & \          & \      & \bm{F}_{k}
        \end{pmatrix}
    \end{equation*}
    称为矩阵 $\bm{A}$ 的有理标准型或 Frobenius 标准型,每个 $\bm{F}_i$ 称为 Frobenius 块。
\end{definition}

\begin{theorem}
    设数域 $\mathbb{K}$ 上的 $n$ 阶矩阵 $\bm{A}$ 的不变因子为
    \[
        1, \ldots, 1, d_{1}(\lambda), \ldots, d_{k}(\lambda)
    \]
    其中 $d_{i}(\lambda) \mid d_{i + 1}(\lambda)(i = 1, 2, \ldots, k - 1)$,则 $\bm{A}$ 的极小多项式 $m(\lambda) = d_{k}(\lambda)$。
\end{theorem}

%——————————————————————————————————%

\section{初等因子}

\begin{definition}
    设 $d_{1}(\lambda), d_{2}(\lambda), \ldots, d_{k}(\lambda)$ 是数域 $\mathbb{K}$ 上矩阵 $\bm{A}$ 的非常数不变因子,在 $\mathbb{K}$ 上把 $d_{i}(\lambda)$ 分解成不可约因式之积:
    \begin{gather}
        d_{1}(\lambda) = p_{1}(\lambda)^{e_{11}}p_{2}(\lambda)^{e_{12}}\cdots p_{t}(\lambda)^{e_{1t}},\notag \\
        d_{2}(\lambda) = p_{1}(\lambda)^{e_{21}}p_{2}(\lambda)^{e_{22}}\cdots p_{t}(\lambda)^{e_{2t}},\notag \\
        \cdots \cdots                                                                                        \\
        d_{k}(\lambda) = p_{1}(\lambda)^{e_{k1}}p_{1}(\lambda)^{e_{k1}}\cdots p_{t}(\lambda)^{e_{kt}},\notag
    \end{gather}\label{equ:7.5.1}
    其中 $e_{ij}$ 是非负整数(注意 $e_{ij}$ 可以为 $0$)。由于 $d_{i}(\lambda) \mid d_{i + 1}(\lambda)$,因此
    \[
        e_{1j} \leqslant e_{2j} \leqslant \cdots \leqslant e_{kj},\quad j = 1, 2, \cdots, t
    \]
    若 $\ref{equ:7.5.1}$ 式中的 $e_{ij} > 0$,则称 $p_{j}(\lambda)^{e_{ij}}$ 为 $\bm{A}$ 的初等因子,$\bm{A}$ 的全体初等因子称为 $\bm{A}$ 的初等因子组。
\end{definition}

\begin{theorem}
    数域 $\mathbb{K}$ 上的两个矩阵 $\bm{A}$ 与 $\bm{B}$ 相似的充分必要条件是它们有相同的初等因子组,即矩阵的初等因子组是矩阵相似关系的全系不变量。
\end{theorem}



%——————————————————————————————————%

\section{Jordan 标准型}

\begin{lemma}
    $r$ 阶矩阵
    \begin{equation}
        \bm{J} = \begin{pmatrix}
            \lambda_0 & 1         & \      & \      & \         \\
            \         & \lambda_0 & 1      & \      & \         \\
            \         & \         & \ddots & \ddots & \         \\
            \         & \         & \      & \ddots & 1         \\
            \         & \         & \      & \      & \lambda_0
        \end{pmatrix}\label{equ:7.6.1}
    \end{equation}
    的初等因子组为 $(\lambda - \lambda_0)^r$。
\end{lemma}

\begin{lemma}
    设特征矩阵 $\lambda \bm{I} - \bm{A}$ 经过初等变换化为下列对角阵:
    \begin{equation*}
        \begin{pmatrix}
            f_{1}(\lambda) & \              & \      & \              \\
            \              & f_{2}(\lambda) & \      & \              \\
            \              & \              & \ddots & \              \\
            \              & \              & \      & f_{n}(\lambda)
        \end{pmatrix}
    \end{equation*}
    其中 $f_{i}(\lambda)(i = 1, 2, \cdots, n)$ 为非零首一多项式。将 $f_{i}(\lambda)$ 作不可约分解,若 $(\lambda - \lambda_0)^k$ 能整除 $f_{i}(\lambda)$,但 $(\lambda - \lambda_0)^{k + 1}$ 不能整除 $f_{i}(\lambda)$,就称 $(\lambda - \lambda_0)^k$ 是 $f_{i}(\lambda)$ 的准素因子,则矩阵 $\bm{A}$ 的初等因子组等于所有 $f_{i}(\lambda)$ 的准素因子组。
\end{lemma}

\begin{lemma}
    设 $\bm{J}$ 是分块对角阵
    \[
        \begin{pmatrix}
            \bm{J}_1 & \        & \      & \        \\
            \        & \bm{J}_2 & \      & \        \\
            \        & \        & \ddots & \        \\
            \        & \        & \      & \bm{J}_k
        \end{pmatrix}
    \]
    其中每个 $\bm{J}_i$ 都是形如 $\ref{equ:7.6.1}$ 的矩阵,$\bm{J}_i$ 的初等因子组为 $(\lambda - \lambda_i)^{r_i}$,则 $\bm{J}$ 的初等因子为
    \[
        (\lambda - \lambda_1)^{r_1}, (\lambda - \lambda_2)^{r_2}, \ldots, (\lambda - \lambda_k)^{r_k}
    \]
\end{lemma}

\begin{theorem}
    设 $\bm{A}$ 是复数域上的矩阵且 $\bm{A}$ 的初等因子为
    \[
        (\lambda - \lambda_1)^{r_1}, (\lambda - \lambda_2)^{r_2}, \ldots, (\lambda - \lambda_k)^{r_k}
    \]
    则 $\bm{A}$ 相似于下列分块对角阵
    \begin{equation}
        \bm{J} = \begin{pmatrix}
            \bm{J}_1 & \        & \      & \        \\
            \        & \bm{J}_2 & \      & \        \\
            \        & \        & \ddots & \        \\
            \        & \        & \      & \bm{J}_k
        \end{pmatrix}
    \end{equation}
    其中 $\bm{J}_i$ 为 $r_i$ 阶矩阵,且
       \begin{equation}
        \bm{J}_i = \begin{pmatrix}
            \lambda_i & 1         & \      & \      & \         \\
            \         & \lambda_i & 1      & \      & \         \\
            \         & \         & \ddots & \ddots & \         \\
            \         & \         & \      & \ddots & 1         \\
            \         & \         & \      & \      & \lambda_i
        \end{pmatrix}
       \end{equation} 
\end{theorem}

\begin{definition}
  我们将矩阵
  \[
        \bm{J} = \begin{pmatrix}
            \bm{J}_1 & \        & \      & \        \\
            \        & \bm{J}_2 & \      & \        \\
            \        & \        & \ddots & \        \\
            \        & \        & \      & \bm{J}_k
        \end{pmatrix}
  \]
  称为 $\bm{A}$ 的 Jordan 标准型,每个 $\bm{J}_i$ 称为 $\bm{A}$ 的 Jordan 块。
\end{definition}

\begin{theorem}
  设 $\varphi$ 是复数域上线性空间 $V$ 上的线性变换,则必定存在 $V$ 的一组基,使得 $\varphi$ 在这组基下的表示矩阵为 Jordan 标准型。
\end{theorem}

\begin{corollary}
  设 $\bm{A}$ 是 $n$ 阶复矩阵,则下列结论等价:
  \begin{enumerate}
    \item $\bm{A}$ 可对角化;
    \item $\bm{A}$ 的极小多项式无重根;
    \item $\bm{A}$ 的初等因子都是一次多项式。
  \end{enumerate}
\end{corollary}

\begin{corollary}
  设 $\varphi$ 是复数域上线性空间 $V$ 上的线性变换,则 $\varphi$ 可对角化当且仅当 $\varphi$ 的极小多项式无重根,当且仅当 $\varphi$ 的初等因子都是一次多项式。
\end{corollary}

\begin{corollary}
  设 $\varphi$ 是复数域上线性空间 $V$ 上的线性变换,$V_0$ 是 $\varphi$ 的不变子空间。若 $\varphi$ 可对角化,则 $\varphi$ 在 $V_0$ 上的限制也可对角化。
\end{corollary}

\begin{corollary}
  设 $\varphi$ 是复数域上线性空间 $V$ 上的线性变换,且 $V = V_1 \oplus V_2 \oplus \cdots \oplus V_k$,其中每个 $V_i$ 都是 $\varphi$ 的不变子空间,则 $\varphi$ 可对角化的充分必要条件是 $\varphi$ 在每个 $V_i$ 上的限制都可对角化。
\end{corollary}

\begin{corollary}
  设 $\bm{A}$ 是数域 $\mathbb{K}$ 上的矩阵,如果 $\bm{A}$ 的特征值全在 $\mathbb{K}$ 中,则 $\bm{A}$ 在 $\mathbb{K}$ 上相似于其 Jordan 标准型。
\end{corollary}

%——————————————————————————————————%

\section{}






%——————————————————————————————————%

\section{}






%——————————————————————————————————%
%\newpage

%\chapter{二次型}

%——————————————————————————————————%

\section{二次型的化简与矩阵的合同}

\begin{definition}{二次型}
    设 $f$ 是数域 $\mathbb{K}$ 上的 $n$ 元二次齐次多项式:
    \begin{align}
        f(x_1, x_2, \ldots, x_n) = & \ a_{11}x_{1}^{2} + 2a_{12}x_{1}x_{2} + \cdots + 2a_{1n}x_{1}x_{n}       \notag \\
                                   & + a_{22}x_{2}^{2} + \cdots + 2a_{2n}x_{2}x_{n} + \cdots + a_{nn}x_{n}^{2}
    \end{align}
    称 $f$ 为 $\mathbb{K}$ 上的 $n$ 元二次型,简称二次型。
\end{definition}

\begin{definition}{矩阵合同}
    设 $\bm{A}, \bm{B}$ 是数域 $\mathbb{K}$ 上的 $n$ 阶矩阵,若存在 $n$ 阶可逆阵 $\bm{C}$,使得
    \[
        \bm{B} = \bm{C}'\bm{AC}
    \]
    则称 $\bm{B}$ 与 $\bm{A}$ 是合同的,或称 $\bm{B}$ 与 $\bm{A}$ 有合同关系。
\end{definition}

\begin{remark}
    合同关系是一个等价关系。
\end{remark}

\begin{lemma}\label{lem:8.1.1}
    对称阵 $\bm{A}$ 的下列变换都是合同变换:
    \begin{enumerate}
        \item 对换 $\bm{A}$ 的第 $i$ 行与第 $j$ 行,再对换第 $i$ 列与第 $j$ 列;
        \item 将非零常数 $k$ 乘以 $\bm{A}$ 的第 $i$ 行,再将 $k$ 乘以第 $i$ 列;
        \item 将 $\bm{A}$ 的第 $i$ 行乘以 $k$ 加到第 $j$ 行上,再将第 $i$ 列乘以 $k$ 加到第 $j$ 列上。
    \end{enumerate}
\end{lemma}

\begin{lemma}
    设 $\bm{A}$ 是数域 $\mathbb{K}$ 上的非零对称阵,则必定存在可逆阵 $\bm{C}$,使 $\bm{C}'\bm{AC}$ 的第 $(1, 1)$ 元素不等于零。
\end{lemma}

\begin{theorem}
    设 $\bm{A}$ 是数域 $\mathbb{K}$ 上的 $n$ 阶对称阵,则必定存在 $\mathbb{K}$ 上的可逆阵 $\bm{C}$,使 $\bm{C}'\bm{AC}$ 为对角阵。
\end{theorem}


%——————————————————————————————————%

\section{二次型的化简}

\begin{proposition}{配方法}
    利用下列公式:
    \begin{align*}
        (x_1 + x_2 + \cdots + x_n)^2 = & \ x_{1}^{2} + x_{2}^{2} + \cdots + x_{n}^{2}       \\
                                       & + 2x_{1}x_{2} + 2x_{1}x_{3} + \cdots + 2x_{1}x_{n} \\
                                       & + 2x_{2}x_{3} + \cdots + 2x_{2}x_{n}               \\
                                       & + \cdots                                           \\
                                       & + 2x_{n - 1}x_{n}
    \end{align*}
\end{proposition}

\begin{proposition}{初等变换法}
    利用引理 $\ref{lem:8.1.1}$
\end{proposition}





%——————————————————————————————————%

\section{}






%——————————————————————————————————%

\section{}






%——————————————————————————————————%

\section{}






%——————————————————————————————————%

\section{}






%——————————————————————————————————%

\section{}






%——————————————————————————————————%

\section{}






%——————————————————————————————————%

\section{}






%——————————————————————————————————%

\section{}






%——————————————————————————————————%

\section{}






%——————————————————————————————————%

\section{}







%\newpage

%\chapter{内积空间}

%——————————————————————————————————%

\section{内积空间}

\begin{definition}
    设 $V$ 是实数域上的线性空间,若对 $V$ 中任意一组有序向量 $\{\bm{\alpha}, \bm{\beta}\}$,都唯一地对应一个实数,记为 $\langle \bm{\alpha}, \bm{\beta} \rangle$,满足下列条件:
    \begin{enumerate}
        \item $\langle \bm{\beta}, \bm{\alpha} \rangle = \langle \bm{\alpha}, \bm{\beta} \rangle$;
        \item $\langle \bm{\alpha} + \bm{\beta}, \bm{\gamma} \rangle = \langle \bm{\alpha}, \bm{\gamma} \rangle + \langle \bm{\beta}, \bm{\gamma} \rangle$;
        \item $\langle c\bm{\alpha}, \bm{\beta} \rangle = c\langle \bm{\alpha}, \bm{\beta} \rangle$,其中 $c$ 为任一实数;
        \item $\langle \bm{\alpha}, \bm{\alpha} \rangle \geqslant 0$ 且等号成立当且仅当 $\bm{\alpha} = \bm{0}$。
    \end{enumerate}
    则称在 $V$ 上定义了一个内积,实数 $\langle \bm{\alpha}, \bm{\beta} \rangle$ 称为 $\bm{\alpha}$ 与 $\bm{\beta}$ 的内积。线性空间 $V$ 称为实内积空间。有限维实内积空间称为 Euclid 空间,简称欧氏空间。
\end{definition}

\begin{definition}
    设 $V$ 是复数域上的线性空间,若对 $V$ 中任意一组有序向量 $\{\bm{\alpha}, \bm{\beta}\}$,都唯一地对应一个复数,记为 $\langle \bm{\alpha}, \bm{\beta} \rangle$,满足下列条件:
    \begin{enumerate}
        \item $\langle \bm{\beta}, \bm{\alpha} \rangle = \overline{\langle \bm{\alpha}, \bm{\beta} \rangle}$;
        \item $\langle \bm{\alpha} + \bm{\beta}, \bm{\gamma} \rangle = \langle \bm{\alpha}, \bm{\gamma} \rangle + \langle \bm{\beta}, \bm{\gamma} \rangle$;
        \item $\langle c\bm{\alpha}, \bm{\beta} \rangle = c\langle \bm{\alpha}, \bm{\beta} \rangle$,其中 $c$ 为任一复数;
        \item $\langle \bm{\alpha}, \bm{\alpha} \rangle \geqslant 0$ 且等号成立当且仅当 $\bm{\alpha} = \bm{0}$。
    \end{enumerate}
    则称在 $V$ 上定义了一个内积,复数 $\langle \bm{\alpha}, \bm{\beta} \rangle$ 称为 $\bm{\alpha}$ 与 $\bm{\beta}$ 的内积。线性空间 $V$ 称为复内积空间。有限维复内积空间称为酉空间。
\end{definition}

\begin{remark}
    由 $(1)$ 和 $(3)$,有 $\langle \bm{\alpha}, c\bm{\beta} \rangle = \overline{c}\langle \bm{\alpha}, \bm{\beta} \rangle$。
\end{remark}

\begin{definition}
    设 $\mathbb{R}_n$ 是 $n$ 维实列向量空间,$\bm{\alpha} = (x_1, x_2, \ldots, x_n)^{\mathrm{T}}, \bm{\beta} = (y_1, y_2, \ldots, y_n)^{\mathrm{T}}$,定义
    \[
        \langle \bm{\alpha}, \bm{\beta} \rangle = x_{1}y_{1} + x_{2}y_{2} + \cdots + x_{n}y_{n}
    \]
    则 $\mathbb{R}_n$ 成为一个欧式空间,上述内积称为 $\mathbb{R}_n$ 的标准内积。
\end{definition}

\begin{definition}
    设 $\mathbb{C}_n$ 是 $n$ 维复列向量空间,$\bm{\alpha} = (x_1, x_2, \ldots, x_n)^{\mathrm{T}}, \bm{\beta} = (y_1, y_2, \ldots, y_n)^{\mathrm{T}}$,定义
    \[
        \langle \bm{\alpha}, \bm{\beta} \rangle = x_{1}\overline{y}_{1} + x_{2}\overline{y}_{2} + \cdots + x_{n}\overline{y}_{n}
    \]
    则 $\mathbb{C}_n$ 成为一个酉空间,上述内积称为 $\mathbb{C}_n$ 的标准内积。
\end{definition}

\begin{proposition}
    设 $V$ 是 $n$ 维实列向量空间,$\bm{G}$ 是 $n$ 阶正定实对称阵,对 $\bm{\alpha}, \bm{\beta} \in V$,定义
    \[
        \langle \bm{\alpha}, \bm{\beta} \rangle = \bm{\alpha}^{\mathrm{T}}\bm{G\beta}
    \]
    则 $V$ 成为一个欧式空间。当 $\bm{G} = \bm{I}_n$,即 $\bm{G}$ 为单位阵时,$V$ 上的内积就是标准内积。

    实列向量空间的标准内积用矩阵乘法可表示为
    \[
        \langle \bm{\alpha}, \bm{\beta} \rangle = \bm{\alpha}^{\mathrm{T}}\bm{\beta}
    \]

    实行向量空间的标准内积用矩阵乘法可表示为
    \[
        \langle \bm{\alpha}, \bm{\beta} \rangle = \bm{\alpha}\bm{\beta}^{\mathrm{T}}
    \]
\end{proposition}

\begin{proposition}
    设 $U$ 是 $n$ 维复列向量空间,$\bm{H}$ 是 $n$ 阶正定 Hermite 阵,对 $\bm{\alpha}, \bm{\beta} \in U$,定义
    \[
        \langle \bm{\alpha}, \bm{\beta} \rangle = \bm{\alpha}^{\mathrm{T}}\bm{H}\overline{\bm{\beta}}
    \]
    则 $U$ 成为一个欧式空间。当 $\bm{H} = \bm{I}_n$,即 $\bm{H}$ 为单位阵时,$U$ 上的内积就是标准内积。

    复列向量空间的标准内积用矩阵乘法可表示为
    \[
        \langle \bm{\alpha}, \bm{\beta} \rangle = \bm{\alpha}^{\mathrm{T}}\overline{\bm{\beta}}
    \]

    复行向量空间的标准内积用矩阵乘法可表示为
    \[
        \langle \bm{\alpha}, \bm{\beta} \rangle = \bm{\alpha}\overline{\bm{\beta}}^{\mathrm{T}}
    \]
\end{proposition}

\begin{definition}
    设 $V$ 是内积空间,$\bm{\alpha}$ 是 $V$ 中的向量,定义 $\bm{\alpha}$ 的范数为
    \[
        \Vert \bm{\alpha} \Vert = \sqrt{\langle \bm{\alpha}, \bm{\alpha} \rangle}
    \]
    即实数 $\langle \bm{\alpha}, \bm{\alpha} \rangle$ 的算术平方根。
\end{definition}

\begin{remark}
    根据范数的定义,$\Vert \bm{\alpha} \Vert = 0$ 当且仅当 $\bm{\alpha} = \bm{0}$。
\end{remark}

\begin{remark}
    当 $V = \mathbb{R}^n$ 且内积为标准内积时,若 $\bm{\alpha} = (x_1, x_2, \ldots, x_n)$,则
    \[
        \Vert \bm{\alpha} \Vert = \sqrt{x_{1}^{2} + x_{2}^{2} + \cdots + x_{n}^{2}}
    \]
\end{remark}

\begin{remark}
    设 $\bm{\alpha}, \bm{\beta} \in V$,则 $\bm{\alpha}$ 与 $\bm{\beta}$ 的距离定义为
    \[
        d(\bm{\alpha}, \bm{\beta}) = \Vert \bm{\alpha} - \bm{\beta} \Vert
    \]
    显然 $d(\bm{\alpha}, \bm{\beta}) = d(\bm{\beta}, \bm{\alpha})$。
\end{remark}

\begin{theorem}
    设 $V$ 是内积空间,$\bm{\alpha}, \bm{\beta} \in V$,$c$ 是任一常数,则
    \begin{enumerate}
        \item $\Vert c \bm{\alpha} \Vert = \vert c \vert \cdot \Vert \bm{\alpha} \Vert $;
        \item $\vert \langle \bm{\alpha}, \bm{\beta} \rangle \vert \leqslant \Vert \bm{\alpha} \Vert \cdot \Vert \bm{\beta} \Vert $;
        \item $\Vert \bm{\alpha} + \bm{\beta} \Vert \leqslant \Vert \bm{\alpha} \Vert + \Vert \bm{\beta} \Vert $。
    \end{enumerate}
\end{theorem}

\begin{remark}
    我们将 $(2)$ 称为 Cauchy-Schuwarz 不等式,$(3)$ 称为三角不等式。
\end{remark}

\begin{definition}
    当 $V$ 是实内积空间时,定义非零向量 $\bm{\alpha}, \bm{\beta}$ 的夹角 $\theta$ 的余弦为
    \[
        \cos \theta = \frac{\langle \bm{\alpha}, \bm{\beta} \rangle}{\Vert \bm{\alpha} \Vert \cdot \Vert \bm{\beta} \Vert}
    \]
    当 $V$ 是复内积空间时,定义非零向量 $\bm{\alpha}, \bm{\beta}$ 的夹角 $\theta$ 的余弦为
    \[
        \cos \theta = \frac{\vert \langle \bm{\alpha}, \bm{\beta} \rangle \vert}{\Vert \bm{\alpha} \Vert \cdot \Vert \bm{\beta} \Vert}
    \]
    若内积空间中的两个向量 $\bm{\alpha}, \bm{\beta}$ 满足 $\langle \bm{\alpha}, \bm{\beta} \rangle = 0$,则称 $\bm{\alpha}$ 与 $\bm{\beta}$ 垂直或正交,记为 $\bm{\alpha} \perp \bm{\beta}$。
\end{definition}

\begin{remark}
    正交的性质:
    \begin{enumerate}
        \item 零向量与任何向量都正交;
        \item 若 $\bm{\alpha}$ 与 $\bm{\beta}$ 正交,则 $\bm{\beta}$ 也与 $\bm{\alpha}$ 正交;
        \item 两个非零向量 $\bm{\alpha}, \bm{\beta}$ 正交时夹角为 $90^{\circ}$。
    \end{enumerate}
\end{remark}

\begin{remark}
    若 $\bm{\alpha}$ 与 $\bm{\beta}$ 正交,则 $\langle \bm{\alpha}, \bm{\beta} \rangle = \langle \bm{\beta}, \bm{\alpha} \rangle = 0$,于是有勾股定理 ${\Vert \bm{\alpha} + \bm{\beta} \Vert}^{2} = {\Vert \bm{\alpha} \Vert}^{2} + {\Vert \bm{\beta} \Vert}^{2}$;
\end{remark}

\begin{remark}
    设 $V$ 是 $n$ 维实行向量空间,内积取标准内积,则得 Cauchy 不等式
    \[
        (x_{1}y_{1} + x_{2}y_{2} + \cdots + x_{n}y_{n})^{2} \leqslant  (x_{1}^{2} + x_{2}^{2} + \cdots + x_{n}^{2})(y_{1}^{2} + y_{2}^{2} + \cdots + y_{n}^{2})
    \]
\end{remark}

\begin{remark}
    设 $V$ 是由 $[a, b]$ 区间上的连续函数全体构成的实线性空间,内积定义为
    \[
        \langle f, g \rangle = \int_{a}^{b}f(t)g(t)\mathrm{d}t
    \]
    则得 Schuwarz 不等式
    \[
        \left(\int_{a}^{b}f(t)g(t)\mathrm{d}t\right)^{2} \leqslant \int_{a}^{b}f(t)^{2}\mathrm{d}t \int_{a}^{b}g(t)^{2}\mathrm{d}t
    \]
\end{remark}
%——————————————————————————————————%

\section{}






%——————————————————————————————————%

\section{}






%——————————————————————————————————%

\section{}






%——————————————————————————————————%

\section{}






%——————————————————————————————————%

\section{}






%——————————————————————————————————%

\section{}






%——————————————————————————————————%

\section{}






%——————————————————————————————————%

\section{}






%——————————————————————————————————%

\section{}






%——————————————————————————————————%

\section{}






%——————————————————————————————————%

\section{}







%\newpage

%\input{10}
%\newpage

% \printbibliography
\end{document}
