\chapter{相似标准型}

%——————————————————————————————————%

\section{多项式矩阵}

\begin{definition}
    若矩阵中每个元素都是以 $\lambda$ 为未定元的数域 $\mathbb{K}$ 上的多项式,则称该矩阵为多项式矩阵或 $\lambda$ -矩阵,记为
    \[
        \bm{A}(\lambda) = \begin{pmatrix}
            a_{11}(\lambda) & a_{12}(\lambda) & \cdots & a_{1n}(\lambda) \\
            a_{21}(\lambda) & a_{22}(\lambda) & \cdots & a_{2n}(\lambda) \\
            \vdots          & \vdots          & \      & \vdots          \\
            a_{m1}(\lambda) & a_{m2}(\lambda) & \cdots & a_{mn}(\lambda) \\
        \end{pmatrix}
    \]
\end{definition}

\begin{definition}
    对 $\lambda$ -矩阵 $\bm{A}(\lambda)$ 施行的下列三种变换称为 $\lambda$ -矩阵的初等行变换:
    \begin{enumerate}
        \item 将 $\bm{A}(\lambda)$ 的两行对换;
        \item 将 $\bm{A}(\lambda)$ 的第 $i$ 行乘以 $\mathbb{K}$ 中的非零常数 $c$;
        \item 将 $\bm{A}(\lambda)$ 的第 $i$ 行乘以 $\mathbb{K}$ 中的多项式 $f(\lambda)$ 后加到第 $j$ 行上去。
    \end{enumerate}
\end{definition}

\begin{definition}
    若 $\bm{A}(\lambda)$ 和 $\bm{B}(\lambda)$ 是同阶 $\lambda$ -矩阵且 $\bm{A}(\lambda)$ 经过 $\lambda$ -矩阵的初等变换后可变为 $\bm{B}(\lambda)$,则称 $\bm{A}(\lambda)$ 与 $\bm{B}(\lambda)$ 相抵。
\end{definition}

\begin{remark}
    $\lambda$ -矩阵的相抵关系也是一种等价关系。
\end{remark}

\begin{definition}
    下列三种矩阵称为初等 $\lambda$ -矩阵:
    \begin{enumerate}
        \item 将 $n$ 阶单位阵的第 $i$ 行与第 $j$ 行对换,记为 $\bm{P}_{ij}$;
        \item 将 $n$ 阶单位阵的第 $i$ 行乘以非零常数 $c$,记为 $\bm{P}_{i}(c)$;
        \item 将 $n$ 阶单位阵的第 $i$ 行乘以多项式 $f(\lambda)$ 后加到第 $j$ 行上去得到的矩阵,记为 $\bm{T}_{ij}(f(\lambda))$。
    \end{enumerate}
\end{definition}

\begin{theorem}
    对 $\lambda$ -矩阵 $\bm{A}(\lambda)$ 施以第 $k(k = 1, 2, 3)$ 类初等行(列)变换等于用第 $k$ 类初等 $\lambda$ -矩阵左(右)乘以 $\bm{A}(\lambda)$。
\end{theorem}

\begin{definition}
    若 $\bm{A}(\lambda)$ 和 $\bm{B}(\lambda)$ 都是 $n$ 阶 $\lambda$ -矩阵,且
    \[
        \bm{A}(\lambda)\bm{B}(\lambda) = \bm{B}(gl)\bm{A}(\lambda) = \bm{I}_{n}
    \]
    则称 $\bm{B}(\lambda)$ 是 $\bm{A}(\lambda)$ 的逆 $\lambda$ -矩阵。这时称 $\bm{A}(\lambda)$ 为可逆 $\lambda$ -矩阵,在不引起混淆的情况下,有时简称为可逆阵。
\end{definition}

\begin{lemma}
    设 $\bm{M}(\lambda)$ 与 $\bm{N}(\lambda)$ 是两个 $n$ 阶 $\lambda$ - 矩阵且都不等于零。又设 $\bm{B}$ 为 $n$ 阶数字矩阵,则必定存在 $\lambda$ -矩阵 $\bm{Q}(\lambda)$ 及 $\bm{S}(\lambda)$ 和数字矩阵 $\bm{R}$ 及 $\bm{T}$,使得下式成立:
    \begin{align*}
        \bm{M}(\lambda) = (\lambda \bm{I} - \bm{B})\bm{Q}(\lambda) + \bm{R} \\
        \bm{N}(\lambda) = \bm{S}(\lambda)(\lambda \bm{I} - \bm{B}) + \bm{T}
    \end{align*}
\end{lemma}

\begin{theorem}
    设 $\bm{A}, \bm{B}$ 是数域 $\mathbb{K}$ 上的矩阵,则 $\bm{A}$ 与 $\bm{B}$ 相似的充分必要条件是 $\lambda$ -矩阵 $\lambda \bm{I} - \bm{A}$ 与 $\lambda \bm{I} - \bm{B}$ 相抵。
\end{theorem}



%——————————————————————————————————%

\section{矩阵的法式}

\begin{lemma}
    设 $\bm{A}(\lambda) = (a_{ij}(\lambda))_{m \times n}$ 是任一非零 $\lambda$ -矩阵,则 $\bm{A}(\lambda)$ 必相抵于这样的一个 $\lambda$ -矩阵 $\bm{B}(\lambda) = (b_{ij}(\lambda))_{m \times n}$,其中 $b_{11}(\lambda) \neq 0$ 且 $b_{11}(\lambda)$ 可整除 $\bm{B}(gl)$ 中的任一元素 $b_{ij}(\lambda)$。
\end{lemma}

\begin{theorem}
    设 $\bm{A}(\lambda)$ 是 $n$ 阶 $\lambda$ -矩阵,则 $\bm{A}(\lambda)$ 相抵于对角阵
    \[
        \operatorname{diag}\{ d_{1}(\lambda), d_{2}(\lambda), \ldots, d_{r}(\lambda); 0, \ldots, 0 \}
    \]
    其中 $d_{i}(\lambda)$ 是非零首一多项式且 $d_{i}(\lambda) \mid d_{i + 1}(\lambda)(i = 1, 2, \ldots, r - 1)$。
\end{theorem}

\begin{corollary}
    任一 $n$ 阶可逆 $\lambda$ -矩阵都可表示为有限个 $\lambda$ -矩阵之积。
\end{corollary}

\begin{corollary}
    设 $\bm{A}$ 是数域 $\mathbb{K}$ 上的 $n$ 阶矩阵,则 $\bm{A}$ 的特征矩阵 $\lambda \bm{I}_{n} - \bm{A}$ 必定相抵于
    \[
        \operatorname{diag}\{ 1, \ldots, 1, d_{1}(\lambda), \ldots, d_{m}(\lambda) \}
    \]
    其中 $d_{i}(\lambda) \mid d_{i + 1}(\lambda)(i = 1, 2, \ldots, m - 1)$。
\end{corollary}


\begin{definition}
    对角 $\lambda$ -矩阵
    \[
        \operatorname{diag}\{ d_{1}(\lambda), d_{2}(\lambda), \ldots, d_{r}(\lambda); 0, \ldots, 0 \}
    \]
    称为 $\bm{A}(\lambda)$ 的法式或相抵标准型。
\end{definition}


%——————————————————————————————————%

\section{不变因子}

\begin{definition}
    设 $\bm{A}(\lambda)$ 是 $n$ 阶 $\lambda$ -矩阵,$k \in \mathbb{N}^*$ 且 $k \leqslant n$。如果 $\bm{A}(\lambda)$ 有一个 $k$ 阶子式不为零,则定义 $\bm{A}(\lambda)$ 的 $k$ 阶行列式因子 $D_{k}(\lambda)$ 为 $\bm{A}(\lambda)$ 的所有 $k$ 阶子式的最大公因式(首一多项式)。如果 $\bm{A}(\lambda)$ 的所有 $k$ 阶子式都等于零,则定义 $\bm{A}(\lambda)$ 的 $k$ 阶行列式因子 $D_{k}(\lambda)$ 为零。
\end{definition}

\begin{lemma}
    设 $D_{1}(\lambda), D_{2}(\lambda), \ldots, D_{r}(\lambda)$ 是 $\bm{A}(\lambda)$ 的非零行列式因子,则
    \[
        D_{i}(\lambda) \mid D_{i + 1}(\lambda),\quad i = 1, 2, \ldots, r - 1
    \]
\end{lemma}

\begin{definition}
    设 $D_{1}(\lambda), D_{2}(\lambda), \ldots, D_{r}(\lambda)$ 是 $\bm{A}(\lambda)$ 的非零行列式因子,则 $g_{1}(\lambda) = D_{1}(\lambda), g_{2}(\lambda) = \dfrac{D_{2}(\lambda)}{D_{1}(\lambda)}, \ldots, g_{r}(\lambda) = \dfrac{D_{r}(\lambda)}{D_{r - 1}(\lambda)}$ 称为 $\bm{A}(\lambda)$ 的不变因子。
\end{definition}

\begin{theorem}
    相抵的 $\lambda$ -矩阵有相同的行列式因子,从而有相同的不变因子。
\end{theorem}

\begin{corollary}
    设 $n$ 阶 $\lambda$ -矩阵 $\bm{A}(\lambda)$ 的法式为
    \[
        \varLambda = \operatorname{diag}\{ d_{1}(\lambda), d_{2}(\lambda), \ldots, d_{r}(\lambda); 0, \ldots, 0 \}
    \]
    其中 $d_{i}(\lambda)$ 是非零首一多项式且 $d_{i}(\lambda) \mid d_{i + 1}(\lambda)(i = 1, 2, \ldots, r - 1)$,则 $\bm{A}(\lambda)$ 的不变因子为
    \[
        d_{1}(\lambda), d_{2}(\lambda), \ldots, d_{r}(\lambda)
    \]
    特别地,法式和不变因子之间相互唯一确定。
\end{corollary}

\begin{corollary}
    设 $\bm{A}(\lambda), \bm{B}(\lambda)$ 为 $n$ 阶 $\lambda$ -矩阵,则 $\bm{A}(\lambda)$ 与 $\bm{B}(\lambda)$ 相抵当且仅当它们有相同的法式。
\end{corollary}

\begin{corollary}
    $n$ 阶 $\lambda$ -矩阵 $\bm{A}(\lambda)$ 的法式与初等变换的选取无关。
\end{corollary}

\begin{theorem}
    数域 $\mathbb{K}$ 上的 $n$ 阶矩阵 $\bm{A}$ 与 $\bm{B}$ 相似的充分必要条件是它们的特征矩阵 $\lambda \bm{I} - \bm{A}$ 与 $\lambda \bm{I} - \bm{B}$ 有相同的行列式因子或不变因子。
\end{theorem}

\begin{corollary}
    设 $\mathbb{F} \subseteq \mathbb{K}$ 是两个数域,$\bm{A}, \bm{B}$ 是 $\mathbb{F}$ 上的两个矩阵,则 $\bm{A}$ 与 $\bm{B}$ 在 $\mathbb{F}$ 上相似的充分必要条件是它们在 $\mathbb{K}$ 上相似。
\end{corollary}


%——————————————————————————————————%

\section{有理标准型}

\begin{lemma}\label{lem:7.4.1}
    设 $r$ 阶矩阵
    \[
        \bm{F} = \begin{pmatrix}
            0      & 1          & 0          & \cdots & 0      \\
            0      & 0          & 1          & \cdots & 0      \\
            \vdots & \vdots     & \vdots     & \      & \vdots \\
            0      & 0          & 0          & \cdots & 1      \\
            -a_{r} & -a_{r - 1} & -a_{r - 2} & \cdots & -a_{1}
        \end{pmatrix}
    \]
    则
    \begin{enumerate}
        \item $\bm{F}$ 的行列式因子为
              \begin{equation}\tag{1}
                  1, \ldots, 1, f(\lambda)
              \end{equation}
              其中共有 $r - 1$ 个 $1$,$f(\lambda) = \lambda^r + a_{1}\lambda^{r - 1} + \cdots + a_{r}$,$\bm{F}$ 的不变因子也由 $(1)$ 式给出。
        \item $\bm{F}$ 的极小多项式等于 $f(\lambda)$。
    \end{enumerate}
\end{lemma}

\begin{lemma}
    设 $\lambda$ -矩阵 $\bm{A}(\lambda)$ 相抵于对角 $\lambda$ -矩阵
    \[
        \operatorname{diag}\{ d_{1}(\lambda), d_{2}(\lambda), \ldots, d_{n}(\lambda) \}
    \]
    $\lambda$ -矩阵 $\bm{B}(\lambda)$ 相抵于对角 $\lambda$ -矩阵
    \[
        \operatorname{diag}\{ d_{1}^{'}(\lambda), d_{2}^{'}(\lambda), \ldots, d_{n}^{'}(\lambda) \}
    \]
    且 $d_{1}^{'}(\lambda), d_{2}^{'}(\lambda), \ldots, d_{n}^{'}(\lambda)$ 是 $d_{1}(\lambda), d_{2}(\lambda), \ldots, d_{n}(\lambda)$ 的置换(即若不计次序,这两组多项式完全相同)。
\end{lemma}

\begin{theorem}
    设 $\bm{A}$ 是数域 $\mathbb{K}$ 上的 $n$ 阶方阵,$\bm{A}$ 的不变因子组为
    \[
        1, \ldots, 1, d_{1}(\lambda), \ldots, d_{k}(\lambda)
    \]
    其中 $\deg d_{i}(\lambda) = m_{i} \geqslant 1$,则 $\bm{A}$ 相似于下列分块对角阵:
    \begin{equation}
        \bm{F} = \begin{pmatrix}
            \bm{F}_{1} & \          & \                   \\
            \          & \bm{F}_{2} & \      & \          \\
            \          & \          & \ddots & \          \\
            \          & \          & \      & \bm{F}_{k}
        \end{pmatrix}\label{equ:7.4.4}
    \end{equation}
    其中 $\bm{F}_i$ 的阶等于 $m_i$,且 $\bm{F}_i$ 是形如 $\ref{lem:7.4.1}$ 中的 $\bm{F}$ 矩阵, $\bm{F}_i$ 的最后一行由 $d_{i}(\lambda)$ 的系数(除首项系数之外)的负值组成。
\end{theorem}

\begin{definition}
    我们将分块对角阵
    \begin{equation*}
        \bm{F} = \begin{pmatrix}
            \bm{F}_{1} & \          & \                   \\
            \          & \bm{F}_{2} & \      & \          \\
            \          & \          & \ddots & \          \\
            \          & \          & \      & \bm{F}_{k}
        \end{pmatrix}
    \end{equation*}
    称为矩阵 $\bm{A}$ 的有理标准型或 Frobenius 标准型,每个 $\bm{F}_i$ 称为 Frobenius 块。
\end{definition}

\begin{theorem}
    设数域 $\mathbb{K}$ 上的 $n$ 阶矩阵 $\bm{A}$ 的不变因子为
    \[
        1, \ldots, 1, d_{1}(\lambda), \ldots, d_{k}(\lambda)
    \]
    其中 $d_{i}(\lambda) \mid d_{i + 1}(\lambda)(i = 1, 2, \ldots, k - 1)$,则 $\bm{A}$ 的极小多项式 $m(\lambda) = d_{k}(\lambda)$。
\end{theorem}

%——————————————————————————————————%

\section{初等因子}

\begin{definition}
    设 $d_{1}(\lambda), d_{2}(\lambda), \ldots, d_{k}(\lambda)$ 是数域 $\mathbb{K}$ 上矩阵 $\bm{A}$ 的非常数不变因子,在 $\mathbb{K}$ 上把 $d_{i}(\lambda)$ 分解成不可约因式之积:
    \begin{gather}
        d_{1}(\lambda) = p_{1}(\lambda)^{e_{11}}p_{2}(\lambda)^{e_{12}}\cdots p_{t}(\lambda)^{e_{1t}},\notag \\
        d_{2}(\lambda) = p_{1}(\lambda)^{e_{21}}p_{2}(\lambda)^{e_{22}}\cdots p_{t}(\lambda)^{e_{2t}},\notag \\
        \cdots \cdots                                                                                        \\
        d_{k}(\lambda) = p_{1}(\lambda)^{e_{k1}}p_{1}(\lambda)^{e_{k1}}\cdots p_{t}(\lambda)^{e_{kt}},\notag
    \end{gather}\label{equ:7.5.1}
    其中 $e_{ij}$ 是非负整数(注意 $e_{ij}$ 可以为 $0$)。由于 $d_{i}(\lambda) \mid d_{i + 1}(\lambda)$,因此
    \[
        e_{1j} \leqslant e_{2j} \leqslant \cdots \leqslant e_{kj},\quad j = 1, 2, \cdots, t
    \]
    若 $\ref{equ:7.5.1}$ 式中的 $e_{ij} > 0$,则称 $p_{j}(\lambda)^{e_{ij}}$ 为 $\bm{A}$ 的初等因子,$\bm{A}$ 的全体初等因子称为 $\bm{A}$ 的初等因子组。
\end{definition}

\begin{theorem}
    数域 $\mathbb{K}$ 上的两个矩阵 $\bm{A}$ 与 $\bm{B}$ 相似的充分必要条件是它们有相同的初等因子组,即矩阵的初等因子组是矩阵相似关系的全系不变量。
\end{theorem}



%——————————————————————————————————%

\section{Jordan 标准型}

\begin{lemma}
    $r$ 阶矩阵
    \begin{equation}
        \bm{J} = \begin{pmatrix}
            \lambda_0 & 1         & \      & \      & \         \\
            \         & \lambda_0 & 1      & \      & \         \\
            \         & \         & \ddots & \ddots & \         \\
            \         & \         & \      & \ddots & 1         \\
            \         & \         & \      & \      & \lambda_0
        \end{pmatrix}\label{equ:7.6.1}
    \end{equation}
    的初等因子组为 $(\lambda - \lambda_0)^r$。
\end{lemma}

\begin{lemma}
    设特征矩阵 $\lambda \bm{I} - \bm{A}$ 经过初等变换化为下列对角阵:
    \begin{equation*}
        \begin{pmatrix}
            f_{1}(\lambda) & \              & \      & \              \\
            \              & f_{2}(\lambda) & \      & \              \\
            \              & \              & \ddots & \              \\
            \              & \              & \      & f_{n}(\lambda)
        \end{pmatrix}
    \end{equation*}
    其中 $f_{i}(\lambda)(i = 1, 2, \cdots, n)$ 为非零首一多项式。将 $f_{i}(\lambda)$ 作不可约分解,若 $(\lambda - \lambda_0)^k$ 能整除 $f_{i}(\lambda)$,但 $(\lambda - \lambda_0)^{k + 1}$ 不能整除 $f_{i}(\lambda)$,就称 $(\lambda - \lambda_0)^k$ 是 $f_{i}(\lambda)$ 的准素因子,则矩阵 $\bm{A}$ 的初等因子组等于所有 $f_{i}(\lambda)$ 的准素因子组。
\end{lemma}

\begin{lemma}
    设 $\bm{J}$ 是分块对角阵
    \[
        \begin{pmatrix}
            \bm{J}_1 & \        & \      & \        \\
            \        & \bm{J}_2 & \      & \        \\
            \        & \        & \ddots & \        \\
            \        & \        & \      & \bm{J}_k
        \end{pmatrix}
    \]
    其中每个 $\bm{J}_i$ 都是形如 $\ref{equ:7.6.1}$ 的矩阵,$\bm{J}_i$ 的初等因子组为 $(\lambda - \lambda_i)^{r_i}$,则 $\bm{J}$ 的初等因子为
    \[
        (\lambda - \lambda_1)^{r_1}, (\lambda - \lambda_2)^{r_2}, \ldots, (\lambda - \lambda_k)^{r_k}
    \]
\end{lemma}

\begin{theorem}
    设 $\bm{A}$ 是复数域上的矩阵且 $\bm{A}$ 的初等因子为
    \[
        (\lambda - \lambda_1)^{r_1}, (\lambda - \lambda_2)^{r_2}, \ldots, (\lambda - \lambda_k)^{r_k}
    \]
    则 $\bm{A}$ 相似于下列分块对角阵
    \begin{equation}
        \bm{J} = \begin{pmatrix}
            \bm{J}_1 & \        & \      & \        \\
            \        & \bm{J}_2 & \      & \        \\
            \        & \        & \ddots & \        \\
            \        & \        & \      & \bm{J}_k
        \end{pmatrix}
    \end{equation}
    其中 $\bm{J}_i$ 为 $r_i$ 阶矩阵,且
    \begin{equation}
        \bm{J}_i = \begin{pmatrix}
            \lambda_i & 1         & \      & \      & \         \\
            \         & \lambda_i & 1      & \      & \         \\
            \         & \         & \ddots & \ddots & \         \\
            \         & \         & \      & \ddots & 1         \\
            \         & \         & \      & \      & \lambda_i
        \end{pmatrix}
    \end{equation}
\end{theorem}

\begin{definition}
    我们将矩阵
    \[
        \bm{J} = \begin{pmatrix}
            \bm{J}_1 & \        & \      & \        \\
            \        & \bm{J}_2 & \      & \        \\
            \        & \        & \ddots & \        \\
            \        & \        & \      & \bm{J}_k
        \end{pmatrix}
    \]
    称为 $\bm{A}$ 的 Jordan 标准型,每个 $\bm{J}_i$ 称为 $\bm{A}$ 的 Jordan 块。
\end{definition}

\begin{theorem}
    设 $\varphi$ 是复数域上线性空间 $V$ 上的线性变换,则必定存在 $V$ 的一组基,使得 $\varphi$ 在这组基下的表示矩阵为 Jordan 标准型。
\end{theorem}

\begin{corollary}
    设 $\bm{A}$ 是 $n$ 阶复矩阵,则下列结论等价:
    \begin{enumerate}
        \item $\bm{A}$ 可对角化;
        \item $\bm{A}$ 的极小多项式无重根;
        \item $\bm{A}$ 的初等因子都是一次多项式。
    \end{enumerate}
\end{corollary}

\begin{corollary}
    设 $\varphi$ 是复数域上线性空间 $V$ 上的线性变换,则 $\varphi$ 可对角化当且仅当 $\varphi$ 的极小多项式无重根,当且仅当 $\varphi$ 的初等因子都是一次多项式。
\end{corollary}

\begin{corollary}
    设 $\varphi$ 是复数域上线性空间 $V$ 上的线性变换,$V_0$ 是 $\varphi$ 的不变子空间。若 $\varphi$ 可对角化,则 $\varphi$ 在 $V_0$ 上的限制也可对角化。
\end{corollary}

\begin{corollary}
    设 $\varphi$ 是复数域上线性空间 $V$ 上的线性变换,且 $V = V_1 \oplus V_2 \oplus \cdots \oplus V_k$,其中每个 $V_i$ 都是 $\varphi$ 的不变子空间,则 $\varphi$ 可对角化的充分必要条件是 $\varphi$ 在每个 $V_i$ 上的限制都可对角化。
\end{corollary}

\begin{corollary}
    设 $\bm{A}$ 是数域 $\mathbb{K}$ 上的矩阵,如果 $\bm{A}$ 的特征值全在 $\mathbb{K}$ 中,则 $\bm{A}$ 在 $\mathbb{K}$ 上相似于其 Jordan 标准型。
\end{corollary}

%——————————————————————————————————%

\section{Jordan 标准型的应用}

\begin{theorem}
    线性变换 $\varphi$ 的特征值 $\lambda_1$ 的度数等于 $\varphi$ 的 Jordan 标准型中属于特征值 $\lambda_1$ 的 Jordan 块的个数,$\lambda_1$ 的重数等于所有属于特征值 $\lambda_1$ 的 Jordan 块的阶数之和。
\end{theorem}

\begin{definition}
    设 $V_0$ 是线性空间 $V$ 的 $r$ 维子空间,$\psi$ 是 $V$ 上的线性变换。若存在 $\bm{\alpha} \in V_0$,使 $\{\bm{\alpha}, \psi(\alpha), \ldots, \psi^{r - 1}(\alpha)\}$ 构成 $V_0$ 的一组基,则称 $V_0$ 为关于线性变换 $\psi$ 的循环子空间。
\end{definition}

\begin{definition}
    设 $\lambda_0$ 是 $n$ 维复线性空间 $V$ 上的线性变换 $\varphi$ 的特征值,则
    \[
        R(\lambda_0) = \{\bm{v} \in V : (\varphi - \lambda_0 \bm{I})^{n}(\bm{v}) = \bm{0}\}
    \]
    构成了 $V$ 的子空间,称为属于特征值 $\lambda_0$ 的根子空间。
\end{definition}

\begin{theorem}
    设 $\varphi$ 是 $n$ 维复线性空间 $V$ 上的线性变换
    \begin{enumerate}
        \item 若 $\varphi$ 的初等因子组为
              \[
                  (\lambda - \lambda_1)^{r_1}, (\lambda - \lambda_2)^{r_2}, \ldots, (\lambda - \lambda_k)^{r_k}
              \]
              则 $V$ 可分解为 $k$ 个不变子空间的直和:
              \begin{equation}
                  V = V_1 \oplus V_2 \oplus \cdots \oplus V_k
              \end{equation}
              其中 $V_i$ 是维数等于 $r_i$ 的关于 $\varphi - \lambda_{i}\bm{I}$ 的循环子空间。
        \item 若 $\lambda_1, \lambda_2, \ldots, \lambda_s$ 是 $\varphi$ 的全体不同特征值,则 $V$ 可分解为 $s$ 个不变子空间的直和:
              \begin{equation}
                  V = R(\lambda_1) \oplus R(\lambda_2) \oplus \cdots \oplus R(\lambda_s)
              \end{equation}
              其中 $R(\lambda_i)$ 是 $\lambda_i$ 的根子空间,$R(\lambda_i)$ 的维数等于 $\lambda_i$ 的重数,且每个 $R(\lambda_i)$ 又可分解为 $(4.6)$ 式中若干个 $V_j$ 的直和。
    \end{enumerate}
\end{theorem}

\begin{lemma}
    设 $\bm{A}, \bm{B}$ 是两个 $n$ 阶可对角化复矩阵且 $\bm{AB} = \bm{BA}$,则它们可同时对角化,即存在可逆阵 $\bm{P}$,使 $\bm{P}^{-1}\bm{AP}$ 和 $\bm{P}^{-1}\bm{BP}$ 都是对角阵。
\end{lemma}

\begin{theorem}{Jordan-Chevalley 分解}
    设 $\bm{A}$ 是 $n$ 阶复矩阵,则 $\bm{A}$ 可分解为 $\bm{A} = \bm{B} + \bm{C}$,其中 $\bm{B}, \bm{C}$ 适合下面的条件:
    \begin{enumerate}
        \item $\bm{B}$ 是可对角化矩阵;
        \item $\bm{C}$ 是幂零阵;
        \item $\bm{BC} = \bm{CB}$;
        \item $\bm{B}, \bm{C}$ 均可表示为 $\bm{A}$ 的多项式。
    \end{enumerate}
    且满足前三个条件的分解是唯一的。
\end{theorem}


%——————————————————————————————————%

\section{}






%——————————————————————————————————%