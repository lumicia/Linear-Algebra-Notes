\chapter{相似标准型}

%——————————————————————————————————%

\section{多项式矩阵}

\begin{definition}
    若矩阵中每个元素都是以 $\lambda$ 为未定元的数域 $\mathbb{K}$ 上的多项式,则称该矩阵为多项式矩阵或 $\lambda$ -矩阵,记为
    \[
        \bm{A}(\lambda) = \begin{pmatrix}
            a_{11}(\lambda) & a_{12}(\lambda) & \cdots & a_{1n}(\lambda) \\
            a_{21}(\lambda) & a_{22}(\lambda) & \cdots & a_{2n}(\lambda) \\
            \vdots          & \vdots          & \      & \vdots          \\
            a_{m1}(\lambda) & a_{m2}(\lambda) & \cdots & a_{mn}(\lambda) \\
        \end{pmatrix}
    \]
\end{definition}

\begin{definition}
    对 $\lambda$ -矩阵 $\bm{A}(\lambda)$ 施行的下列三种变换称为 $\lambda$ -矩阵的初等行变换:
    \begin{enumerate}
        \item 将 $\bm{A}(\lambda)$ 的两行对换;
        \item 将 $\bm{A}(\lambda)$ 的第 $i$ 行乘以 $\mathbb{K}$ 中的非零常数 $c$;
        \item 将 $\bm{A}(\lambda)$ 的第 $i$ 行乘以 $\mathbb{K}$ 中的多项式 $f(\lambda)$ 后加到第 $j$ 行上去。
    \end{enumerate}
\end{definition}

\begin{definition}
    若 $\bm{A}(\lambda)$ 和 $\bm{B}(\lambda)$ 是同阶 $\lambda$ -矩阵且 $\bm{A}(\lambda)$ 经过 $\lambda$ -矩阵的初等变换后可变为 $\bm{B}(\lambda)$,则称 $\bm{A}(\lambda)$ 与 $\bm{B}(\lambda)$ 相抵。
\end{definition}

\begin{remark}
    $\lambda$ -矩阵的相抵关系也是一种等价关系。
\end{remark}

\begin{definition}
    下列三种矩阵称为初等 $\lambda$ -矩阵:
    \begin{enumerate}
        \item 将 $n$ 阶单位阵的第 $i$ 行与第 $j$ 行对换,记为 $\bm{P}_{ij}$;
        \item 将 $n$ 阶单位阵的第 $i$ 行乘以非零常数 $c$,记为 $\bm{P}_{i}(c)$;
        \item 将 $n$ 阶单位阵的第 $i$ 行乘以多项式 $f(\lambda)$ 后加到第 $j$ 行上去得到的矩阵,记为 $\bm{T}_{ij}(f(\lambda))$。
    \end{enumerate}
\end{definition}

\begin{theorem}
    对 $\lambda$ -矩阵 $\bm{A}(\lambda)$ 施以第 $k(k = 1, 2, 3)$ 类初等行(列)变换等于用第 $k$ 类初等 $\lambda$ -矩阵左(右)乘以 $\bm{A}(\lambda)$。
\end{theorem}

\begin{definition}
    若 $\bm{A}(\lambda)$ 和 $\bm{B}(\lambda)$ 都是 $n$ 阶 $\lambda$ -矩阵,且
    \[
        \bm{A}(\lambda)\bm{B}(\lambda) = \bm{B}(gl)\bm{A}(\lambda) = \bm{I}_{n}
    \]
    则称 $\bm{B}(\lambda)$ 是 $\bm{A}(\lambda)$ 的逆 $\lambda$ -矩阵。这时称 $\bm{A}(\lambda)$ 为可逆 $\lambda$ -矩阵,在不引起混淆的情况下,有时简称为可逆阵。
\end{definition}

\begin{lemma}
    设 $\bm{M}(\lambda)$ 与 $\bm{N}(\lambda)$ 是两个 $n$ 阶 $\lambda$ - 矩阵且都不等于零。又设 $\bm{B}$ 为 $n$ 阶数字矩阵,则必定存在 $\lambda$ -矩阵 $\bm{Q}(\lambda)$ 及 $\bm{S}(\lambda)$ 和数字矩阵 $\bm{R}$ 及 $\bm{T}$,使得下式成立:
    \begin{align*}
        \bm{M}(\lambda) = (\lambda \bm{I} - \bm{B})\bm{Q}(\lambda) + \bm{R} \\
        \bm{N}(\lambda) = \bm{S}(\lambda)(\lambda \bm{I} - \bm{B}) + \bm{T}
    \end{align*}
\end{lemma}

\begin{theorem}
    设 $\bm{A}, \bm{B}$ 是数域 $\mathbb{K}$ 上的矩阵,则 $\bm{A}$ 与 $\bm{B}$ 相似的充分必要条件是 $\lambda$ -矩阵 $\lambda \bm{I} - \bm{A}$ 与 $\lambda \bm{I} - \bm{B}$ 相抵。
\end{theorem}



%——————————————————————————————————%

\section{矩阵的法式}

\begin{lemma}
    设 $\bm{A}(\lambda) = (a_{ij}(\lambda))_{m \times n}$ 是任一非零 $\lambda$ -矩阵,则 $\bm{A}(\lambda)$ 必相抵于这样的一个 $\lambda$ -矩阵 $\bm{B}(\lambda) = (b_{ij}(\lambda))_{m \times n}$,其中 $b_{11}(\lambda) \neq 0$ 且 $b_{11}(\lambda)$ 可整除 $\bm{B}(gl)$ 中的任一元素 $b_{ij}(\lambda)$。
\end{lemma}

\begin{theorem}
    设 $\bm{A}(\lambda)$ 是 $n$ 阶 $\lambda$ -矩阵,则 $\bm{A}(\lambda)$ 相抵于对角阵
    \[
        \operatorname{diag}\{ d_{1}(\lambda), d_{2}(\lambda), \cdots, d_{r}(\lambda); 0, \cdots, 0 \}
    \]
    其中 $d_{i}(\lambda)$ 是非零首一多项式且 $d_{i}(\lambda) \mid d_{i + 1}(\lambda)(i = 1, 2, \cdots, r - 1)$。
\end{theorem}

\begin{corollary}
    任一 $n$ 阶可逆 $\lambda$ -矩阵都可表示为有限个 $\lambda$ -矩阵之积。
\end{corollary}

\begin{corollary}
    设 $\bm{A}$ 是数域 $\mathbb{K}$ 上的 $n$ 阶矩阵,则 $\bm{A}$ 的特征矩阵 $\lambda \bm{I}_{n} - \bm{A}$ 必定相抵于
    \[
        \operatorname{diag}\{ 1, \cdots, 1, d_{1}(\lambda), \cdots, d_{m}(\lambda) \}
    \]
    其中 $d_{i}(\lambda) \mid d_{i + 1}(\lambda)(i = 1, 2, \cdots, m - 1)$。
\end{corollary}


\begin{definition}
    对角 $\lambda$ -矩阵
    \[
        \operatorname{diag}\{ d_{1}(\lambda), d_{2}(\lambda), \cdots, d_{r}(\lambda); 0, \cdots, 0 \}
    \]
    称为 $\bm{A}(\lambda)$ 的法式或相抵标准型。
\end{definition}


%——————————————————————————————————%

\section{不变因子}

\begin{definition}
    设 $\bm{A}(\lambda)$ 是 $n$ 阶 $\lambda$ -矩阵,$k \in \mathbb{N}^*$ 且 $k \leqslant n$。如果 $\bm{A}(\lambda)$ 有一个 $k$ 阶子式不为零,则定义 $\bm{A}(\lambda)$ 的 $k$ 阶行列式因子 $D_{k}(\lambda)$ 为 $\bm{A}(\lambda)$ 的所有 $k$ 阶子式的最大公因式(首一多项式)。如果 $\bm{A}(\lambda)$ 的所有 $k$ 阶子式都等于零,则定义 $\bm{A}(\lambda)$ 的 $k$ 阶行列式因子 $D_{k}(\lambda)$ 为零。
\end{definition}

\begin{lemma}
    设 $D_{1}(\lambda), D_{2}(\lambda), \cdots, D_{r}(\lambda)$ 是 $\bm{A}(\lambda)$ 的非零行列式因子,则
    \[
        D_{i}(\lambda) \mid D_{i + 1}(\lambda),\quad i = 1, 2, \cdots, r - 1
    \]
\end{lemma}

\begin{definition}
  设 $D_{1}(\lambda), D_{2}(\lambda), \cdots, D_{r}(\lambda)$ 是 $\bm{A}(\lambda)$ 的非零行列式因子,则 $g_{1}(\lambda) = D_{1}(\lambda), g_{2}(\lambda) = \dfrac{D_{2}(\lambda)}{D_{1}(\lambda)}, \cdots, g_{r}(\lambda) = \dfrac{D_{r}(\lambda)}{D_{r - 1}(\lambda)}$ 称为 $\bm{A}(\lambda)$ 的不变因子。
\end{definition}

\begin{theorem}
  相抵的 $\lambda$ -矩阵有相同的行列式因子,从而有相同的不变因子。
\end{theorem}

\begin{corollary}
  设 $n$ 阶 $\lambda$ -矩阵 $\bm{A}(\lambda)$ 的法式为
  \[
    \varLambda = \operatorname{diag}\{ d_{1}(\lambda), d_{2}(\lambda), \cdots, d_{r}(\lambda); 0, \cdots, 0 \}
    \]
    其中 $d_{i}(\lambda)$ 是非零首一多项式且 $d_{i}(\lambda) \mid d_{i + 1}(\lambda)(i = 1, 2, \cdots, r - 1)$,则 $\bm{A}(\lambda)$ 的不变因子为 $d_{1}(\lambda), d_{2}(\lambda), \cdots, d_{r}(\lambda)$。特别地,法式和不变因子之间相互唯一确定。
\end{corollary}

\begin{corollary}
  设 $\bm{A}(\lambda), \bm{B}(\lambda)$ 为 $n$ 阶 $\lambda$ -矩阵,则 $\bm{A}(\lambda)$ 与 $\bm{B}(\lambda)$ 相抵当且仅当它们有相同的法式。
\end{corollary}

\begin{corollary}
   $n$ 阶 $\lambda$ -矩阵 $\bm{A}(\lambda)$ 的法式与初等变换的选取无关。
\end{corollary}

\begin{theorem}
  数域 $\mathbb{K}$ 上 $n$ 阶矩阵 $\bm{A}$ 与 $\bm{B}$ 相似的充分必要条件是它们的特征矩阵 $\lambda \bm{I} - \bm{A}$ 与 $\lambda \bm{I} - \bm{B}$ 有相同的行列式因子或不变因子。
\end{theorem}

\begin{corollary}
  设 $\mathbb{F} \subseteq \mathbb{K}$ 是两个数域,$\bm{A}, \bm{B}$ 是 $\mathbb{F}$ 上的两个矩阵,则 $\bm{A}$ 与 $\bm{B}$ 在 $\mathbb{F}$ 上相似的充分必要条件是它们在 $\mathbb{K}$ 上相似。
\end{corollary}


%——————————————————————————————————%

\section{}






%——————————————————————————————————%

\section{}






%——————————————————————————————————%

\section{}






%——————————————————————————————————%

\section{}






%——————————————————————————————————%

\section{}






%——————————————————————————————————%

\section{}






%——————————————————————————————————%

\section{}






%——————————————————————————————————%

\section{}






%——————————————————————————————————%

\section{}






