\chapter{二次型}

%——————————————————————————————————%

\section{二次型的化简与矩阵的合同}

\begin{definition}{二次型}
    设 $f$ 是数域 $\mathbb{K}$ 上的 $n$ 元二次齐次多项式:
    \begin{align}
        f(x_1, x_2, \ldots, x_n) = & \ a_{11}x_{1}^{2} + 2a_{12}x_{1}x_{2} + \cdots + 2a_{1n}x_{1}x_{n}       \notag \\
                                   & + a_{22}x_{2}^{2} + \cdots + 2a_{2n}x_{2}x_{n} + \cdots + a_{nn}x_{n}^{2}
    \end{align}
    称 $f$ 为 $\mathbb{K}$ 上的 $n$ 元二次型,简称二次型。
\end{definition}

\begin{definition}{矩阵合同}
    设 $\bm{A}, \bm{B}$ 是数域 $\mathbb{K}$ 上的 $n$ 阶矩阵,若存在 $n$ 阶可逆阵 $\bm{C}$,使得
    \[
        \bm{B} = \bm{C}'\bm{AC}
    \]
    则称 $\bm{B}$ 与 $\bm{A}$ 是合同的,或称 $\bm{B}$ 与 $\bm{A}$ 有合同关系。
\end{definition}

\begin{remark}
    合同关系是一个等价关系。
\end{remark}

\begin{lemma}\label{lem:8.1.1}
    对称阵 $\bm{A}$ 的下列变换都是合同变换:
    \begin{enumerate}
        \item 对换 $\bm{A}$ 的第 $i$ 行与第 $j$ 行,再对换第 $i$ 列与第 $j$ 列;
        \item 将非零常数 $k$ 乘以 $\bm{A}$ 的第 $i$ 行,再将 $k$ 乘以第 $i$ 列;
        \item 将 $\bm{A}$ 的第 $i$ 行乘以 $k$ 加到第 $j$ 行上,再将第 $i$ 列乘以 $k$ 加到第 $j$ 列上。
    \end{enumerate}
\end{lemma}

\begin{lemma}
    设 $\bm{A}$ 是数域 $\mathbb{K}$ 上的非零对称阵,则必定存在可逆阵 $\bm{C}$,使 $\bm{C}'\bm{AC}$ 的第 $(1, 1)$ 元素不等于零。
\end{lemma}

\begin{theorem}
    设 $\bm{A}$ 是数域 $\mathbb{K}$ 上的 $n$ 阶对称阵,则必定存在 $\mathbb{K}$ 上的可逆阵 $\bm{C}$,使 $\bm{C}'\bm{AC}$ 为对角阵。
\end{theorem}


%——————————————————————————————————%

\section{二次型的化简}

\begin{proposition}{配方法}
    利用下列公式:
    \begin{align*}
        (x_1 + x_2 + \cdots + x_n)^2 = & \ x_{1}^{2} + x_{2}^{2} + \cdots + x_{n}^{2}       \\
                                       & + 2x_{1}x_{2} + 2x_{1}x_{3} + \cdots + 2x_{1}x_{n} \\
                                       & + 2x_{2}x_{3} + \cdots + 2x_{2}x_{n}               \\
                                       & + \cdots                                           \\
                                       & + 2x_{n - 1}x_{n}
    \end{align*}
\end{proposition}

\begin{proposition}{初等变换法}
    利用引理 $\ref{lem:8.1.1}$
\end{proposition}





%——————————————————————————————————%

\section{}






%——————————————————————————————————%

\section{}






%——————————————————————————————————%

\section{}






%——————————————————————————————————%

\section{}






%——————————————————————————————————%

\section{}






%——————————————————————————————————%

\section{}






%——————————————————————————————————%

\section{}






%——————————————————————————————————%

\section{}






%——————————————————————————————————%

\section{}






%——————————————————————————————————%

\section{}






