\chapter{二次型}

%——————————————————————————————————%

\section{二次型的化简与矩阵的合同}

\begin{definition}{二次型}
    设 $f$ 是数域 $\mathbb{K}$ 上的 $n$ 元二次齐次多项式:
    \begin{align}
        f(x_1, x_2, \ldots, x_n) = & \ a_{11}x_{1}^{2} + 2a_{12}x_{1}x_{2} + \cdots + 2a_{1n}x_{1}x_{n}       \notag \\
                                   & + a_{22}x_{2}^{2} + \cdots + 2a_{2n}x_{2}x_{n} + \cdots + a_{nn}x_{n}^{2}
    \end{align}
    称 $f$ 为 $\mathbb{K}$ 上的 $n$ 元二次型,简称二次型。
\end{definition}

\begin{definition}{矩阵合同}
    设 $\bm{A}, \bm{B}$ 是数域 $\mathbb{K}$ 上的 $n$ 阶矩阵,若存在 $n$ 阶可逆阵 $\bm{C}$,使得
    \[
        \bm{B} = \bm{C}^{\mathrm{T}}\bm{AC}
    \]
    则称 $\bm{B}$ 与 $\bm{A}$ 是合同的(congruent),或称 $\bm{B}$ 与 $\bm{A}$ 有合同关系(congruence)。
\end{definition}

\begin{remark}
    合同关系是一个等价关系。
\end{remark}

\begin{lemma}\label{lem:8.1.1}
    对称阵 $\bm{A}$ 的下列变换都是合同变换:
    \begin{enumerate}
        \item 对换 $\bm{A}$ 的第 $i$ 行与第 $j$ 行,再对换第 $i$ 列与第 $j$ 列;
        \item 将非零常数 $k$ 乘以 $\bm{A}$ 的第 $i$ 行,再将 $k$ 乘以第 $i$ 列;
        \item 将 $\bm{A}$ 的第 $i$ 行乘以 $k$ 加到第 $j$ 行上,再将第 $i$ 列乘以 $k$ 加到第 $j$ 列上。
    \end{enumerate}
\end{lemma}

\begin{lemma}
    设 $\bm{A}$ 是数域 $\mathbb{K}$ 上的非零对称阵,则必定存在可逆阵 $\bm{C}$,使 $\bm{C}^{\mathrm{T}}\bm{AC}$ 的第 $(1, 1)$ 元素不等于零。
\end{lemma}

\begin{theorem}
    设 $\bm{A}$ 是数域 $\mathbb{K}$ 上的 $n$ 阶对称阵,则必定存在 $\mathbb{K}$ 上的可逆阵 $\bm{C}$,使 $\bm{C}^{\mathrm{T}}\bm{AC}$ 为对角阵。
\end{theorem}


%——————————————————————————————————%

\section{二次型的化简}

\begin{proposition}{配方法}
    利用下列公式:
    \begin{align*}
        (x_1 + x_2 + \cdots + x_n)^2 = & \ x_{1}^{2} + x_{2}^{2} + \cdots + x_{n}^{2}       \\
                                       & + 2x_{1}x_{2} + 2x_{1}x_{3} + \cdots + 2x_{1}x_{n} \\
                                       & + 2x_{2}x_{3} + \cdots + 2x_{2}x_{n}               \\
                                       & + \cdots                                           \\
                                       & + 2x_{n - 1}x_{n}
    \end{align*}
\end{proposition}

\begin{proposition}{初等变换法}
    利用引理 $\ref{lem:8.1.1}$
\end{proposition}





%——————————————————————————————————%

\section{惯性定理}

\begin{definition}
    设 $f(x_1, x_2, \ldots, x_n)$ 是 $n$ 元实二次型,且 $f$ 可化为两个标准型:
    \begin{align*}
        c_{1}y_{1}^{2} + \cdots + c_{p}y_{p}^{2} - c_{p + 1}y_{p + 1}^{2} - \cdots - c_{r}y_{r}^{2} \\
        d_{1}z_{1}^{2} + \cdots + d_{k}z_{k}^{2} - d_{k + 1}z_{k + 1}^{2} - \cdots - d_{r}z_{r}^{2}
    \end{align*}
    其中 $c_{i} > 0, d_{i} > 0$,则必定有 $p = k$。
\end{definition}

\begin{definition}
    设 $f(x_1, x_2, \ldots, x_n)$ 是一个实二次型,且 $f$ 可化为规范标准型
    \[
        f = y_{1}^{2} + \cdots + y_{p}^{2} - y_{p + 1}^{2} - \cdots - y_{r}^{2}
    \]
    的形状,则称 $r$ 是该二次型的秩,$p$ 是它的正惯性指数,$q = r - p$ 是它的负惯性指数,$s = p - q$ 称为 $f$ 的符号差。
\end{definition}

\begin{theorem}
    秩与符号差(或正负惯性指数)是实对称阵在合同关系下的全系不变量。
\end{theorem}

\begin{theorem}
    秩是复对称阵唯一的全系不变量。
\end{theorem}

%——————————————————————————————————%

\section{正定阵与正定矩阵}

\begin{definition}
    设 $f(x_1, x_2, \ldots, x_n) = \bm{x}^{\mathrm{T}}\bm{Ax}$ 是 $n$ 元实二次型,$\bm{A}$ 是相伴矩阵。
    \begin{enumerate}
        \item 若对任意 $n$ 维非零列向量 $\bm{\alpha}$ 均有 $\bm{\alpha}^{\mathrm{T}}\bm{A\alpha} > 0$,则称 $f$ 是正定二次型(简称正定型),矩阵 $\bm{A}$ 称为正定矩阵(简称正定阵);
        \item 若对任意 $n$ 维非零列向量 $\bm{\alpha}$ 均有 $\bm{\alpha}^{\mathrm{T}}\bm{A\alpha} < 0$,则称 $f$ 是负定二次型(简称负定型),矩阵 $\bm{A}$ 称为负定矩阵(简称负定阵);
        \item 若对任意 $n$ 维非零列向量 $\bm{\alpha}$ 均有 $\bm{\alpha}^{\mathrm{T}}\bm{A\alpha} \geqslant 0$,则称 $f$ 是半正定二次型(简称半正定型),矩阵 $\bm{A}$ 称为半正定矩阵(简称半正定阵);
        \item 若对任意 $n$ 维非零列向量 $\bm{\alpha}$ 均有 $\bm{\alpha}^{\mathrm{T}}\bm{A\alpha} \leqslant 0$,则称 $f$ 是半负定二次型(简称半负定型),矩阵 $\bm{A}$ 称为半负定矩阵(简称半负定阵);
        \item 若存在 $\bm{\alpha}, \bm{\beta}$,使 $\bm{\alpha}^{\mathrm{T}}\bm{A\alpha} > 0, \bm{\beta}^{\mathrm{T}}\bm{A\beta} < 0$,则称 $f$ 为不定型。
    \end{enumerate}
\end{definition}

\begin{theorem}
    设 $f(x_1, x_2, \ldots, x_n)$ 是 $n$ 元实二次型,则
    \begin{enumerate}
        \item $f$ 是正定型的充分必要条件是 $f$ 的正惯性指数等于 $n$;
        \item $f$ 是负定型的充分必要条件是 $f$ 的负惯性指数等于 $n$;
        \item $f$ 是半正定型的充分必要条件是 $f$ 的正惯性指数等于 $f$ 的秩 $r$;
        \item $f$ 是半负定型的充分必要条件是 $f$ 的负惯性指数等于 $f$ 的秩 $r$;
    \end{enumerate}
\end{theorem}

\begin{theorem}
    设 $n$ 阶实对称阵 $\bm{A}$,则
    \begin{enumerate}
        \item $\bm{A}$ 是正定阵当且仅当它合同于单位阵 $\bm{I}_n$;
        \item $\bm{A}$ 是负定阵当且仅当它合同于单位阵 $-\bm{I}_n$;
        \item $\bm{A}$ 是半正定阵当且仅当 $\bm{A}$ 合同于下列对角阵
              \[
                  \begin{pmatrix}
                      \bm{I}_{r} & \bm{O} \\
                      \bm{O}     & \bm{O}
                  \end{pmatrix}
              \]
        \item $\bm{A}$ 是半负定阵当且仅当 $\bm{A}$ 合同于下列对角阵
              \[
                  \begin{pmatrix}
                      -\bm{I}_{r} & \bm{O} \\
                      \bm{O}     & \bm{O}
                  \end{pmatrix}
              \]
    \end{enumerate}
\end{theorem}

\begin{definition}
  设 $\bm{A} = (a_{ij})$ 是 $n$ 阶矩阵,$\bm{A}$ 的 $n$ 个子式
  \[
      \begin{vmatrix} 
          a_{11}& a_{12}& \cdots & a_{1k}\\ 
          a_{21}& a_{22}& \cdots & a_{2k}\\ 
          \vdots& \vdots& \ & \vdots \\
          a_{k1}& a_{k2}& \cdots & a_{kk} 
      \end{vmatrix} 
      (k = 1, 2, \ldots, n)
  \]
  称为 $\bm{A}$ 的顺序主子式。
\end{definition}

\begin{theorem}
    $n$ 阶实对称阵 $\bm{A}$ 是正定阵的充分必要条件是它的 $n$ 个顺序主子式全大于零。
\end{theorem}

%——————————————————————————————————%

\section{}






%——————————————————————————————————%

\section{}






%——————————————————————————————————%

\section{}






%——————————————————————————————————%

\section{}






%——————————————————————————————————%

\section{}






%——————————————————————————————————%

\section{}






%——————————————————————————————————%

\section{}






%——————————————————————————————————%

\section{}






